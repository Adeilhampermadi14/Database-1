\documentclass{article}
\usepackage[utf8]{inputenc}


\date{February 2020}

\begin{document}
\title{\huge\textbf{DataBase}}
\date{}

\maketitle
\begin{center}
\vspace{4cm}
Fanny Devita Inggarini\\
D4 TI 1A\\
1.19.40.15\\
\vspace{4cm}
\textbf{PROGRAM DIPLOMA IV TEKNIK INFORMATIKA} \linebreak
\textbf{POLITEKNIK POS INDONESIA} \linebreak
\textbf{BANDUNG}\linebreak
\textbf{2020}\\

\end{center}
\newpage

\section{DataBase}
\par Basis data (database) adalah kumpulan informasi yang terdapat pada 1 tempat dan dapat berupa angka,huruf,gambar ataupun symbol. Data data tersebut biasanya diambil berdasarkan fakta  dan harus jelas darimana  sumbernya tersebut dan data data tersebut  saling berintraksi.
\vspace{0,5cm}
\par Dalam database sendiri ada proses yang disebut dengan yang namanya normalisasi. Dimana, di dalam databse normalisasi ituberfungsi untuk mengindari redudency atau yang biasa di sebut penggandaan data. Jadi, di dalam datase itu sendiri, untuk melakukan penghindaran data ganda maka dilakukannya normalisasi.
\vspace{0,5cm}
\par Dengan adanya Database dapat memecahkan masalah penyimpanan data tradisional yang umumnya menggunakan kertas.  Karena jika menggunakan  data dengan kertas  maka akan memerlukan banyak ruang penyimpanan dan memerlukan banyak biaya.
\vspace{0,5cm}
\par Memudahkan pengaksesan data, penyimpanan data dan mengedit maupun menghapus data. Jadi kita dapat dengan mudah memasukkan data,  serta dapat mengubah atau mengupdate  data . Dengan adanya pengelompokan data  maka interaksi antar dat data dapat dengan mudah  dan lebih cepat kita untuk mencari data data tersebut .



\end{document}
