\documentclass{article}
\usepackage[utf8]{inputenc}

\title{Tugas Database1}
\author{Nama : Petrolina Anastasia Gatto \\ \\Kelas : D4TI1A \\ \\NPM   : 1194030 \\ \\ Dosen Pengampu : Syafirial Fachri Pane, S.T., M.T.I.,EBDP}
\date{Kamis, 27 Februari 2020}

\usepackage{natbib}
\usepackage{graphicx}

\begin{document}

\maketitle

\section{Database}
  \begin{itemize}
      \item Database adalah kumpulan data atau informasi yang terdiri dari angka, simbol,huruf, dan gambar yang dukumpulkan dan dikelompokan didalam suatu software, yang disusun berdasarkan fakta dan saling berelasi.
      \item Dasar database adalah matematika seperti himpunan
      \item Database dibuat agar tidak ada data yang sama atau ganda (Redudansi).
      \item Hal penting yang harus diperhatikan dalam membuat database yaitu harus menampung informasi berdasarkan fakta, sehingga informasi yang diterima valid.
  \end{itemize}
      
\section{Perbedaan Database Dahulu dan Sekarang}
\begin{itemize}
    \item Zaman Dahulu
    \paragraph{}
     Pada zaman dahulu database sudah diterapkan dengan cara mengelompokan benda-benda atau data-data dan menyimpannya dalam sebuah tempat. Contoh :lemari,dompet,dll.
     \paragraph{}
     Zaman dahulu semakin banyak data yang disimpan maka semakin banyak tempat penyimpanan yang dibutuhkan dan membutuhkan waktu yang lama untuk mencari data yang kita butuhkan. 
\end{itemize}

\begin{itemize}
    \item Sekarang
    \paragraph{}
     Menyimpan data atau informasi lebih mudah, yaitu dalam software. Contoh :MySQL,phpMyAdmin,mariaDB,dll.
     
\end{itemize}

\section{Tujuan Database}
\begin{itemize}
    \item Membantu seseorang dalam menyimpan informasi atau data dengan mudah, agar lebih efektif dan efisien (waktu dan tempat).
\end{itemize}

\end{document}
