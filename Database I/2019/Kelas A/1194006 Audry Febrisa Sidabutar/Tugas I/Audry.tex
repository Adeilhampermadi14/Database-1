\documentclass{article}
\usepackage[utf8]{inputenc}

\title{TUGAS MATA KULIAH BASIS DATA I}
\author{Nama : Audry Febrisa Sidabutar \\ Kelas : D4TI - 1A \\ Dosen Pengampu :  Syafrial Fachri Pane, ST., MTI., EBDP.}
\date{Kamis, 27 Februari 2020}

\usepackage{natbib}
\usepackage{graphicx}

\begin{document}

\maketitle
\section{Pengertian Basis Data}
\paragraph{}
Basis Data terdiri dari kata Basis dan juga Data. Basis merupakan gudang dan Data merupakan kumpulan fakta yang mewakili objek.
\paragraph{}
Basis Data atau Data Base merupakan sekumpulan dari beberapa data yang biasanya berupa huruf, angka, gambar maupun simbol-simbol yang dikumpulkan di dalam suatu tempat yang datanya harus saling berelasi dan tersusun sesuai dengan fakta.

\section{Fungsi Basis Data}
\begin{itemize}
    \item Menghindari data-data yang berulang
    \item Memudahkan penyimpanan data, mengedit ataupun menghapus data
    \item Untuk mempermudah menyimpan data secara terstruktur
    \item Untuk mempermudah identifikasi data
\end{itemize}{}

\newpage
\section{Ciri-ciri Basis Data}
\begin{itemize}
    \item Memiliki duplikasi atau replikasi disetiap data
    \item Servernya saling terhubung dan dapat diakses oleh server lain
    \item Memiliki peningkatan performa
\end{itemize}{}

\section{Jenis-jenis Basis Data}
\begin{itemize}
    \item Basis Data Individual : Basis data yang dipakai oleh perseorangan yang biasanya untuk kepentingan pribadi
    \item Basis Data Perusahaan : Basis Data yang dipakai untuk mengakses sejumlah karyawan di dalam sebuah perusahaan
    \item Basis Data Terdistribusi : Basis data yang dipakai dan disimpan pada sejumlah komputer untuk melayani transaksi yang berifat online
    \item Basis Data Publik : Basis data yang bersifat publik, bisa dipakai oleh siapa saja secara gratis
\end{itemize}{}

\newpage
\section{Komponen Basis Data}
\begin{itemize}
    \item Hardware : Perangkat keras(hardware) pada umumnya yaitu komputer, flashdisk, hard disk,dll.
    \item Operating System : untuk mengaktifkan sistem dan operasi dasar komputer seperti windows,linux.
    \item Management System (DBMS) : software yang berguna untuk mengontrol dan mengelola data.
    \item Database : Sekumpulan beberapa data yang terstruktur dengan baik.
    \item User : Komponen yang berinteraksi langsung dengan database.
\end{itemize}{}

\end{document}