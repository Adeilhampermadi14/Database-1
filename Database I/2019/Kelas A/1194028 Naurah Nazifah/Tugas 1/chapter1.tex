\chapter{Definisi Basis Data}

\paragraph{} Pengertian basis data adalah kumpulan data yang terdapat dalam bentuk angka,huruf,gambar,simbol yang dikumpulkan dalam suatu tempat yang dimana datanya tersusun berdasarkan
fakta, dan data tersebut harus saling berelasi untuk mengelompokkan data secara sistematis agar terhindar dari duplikasi.

Contoh DBMS(Software) :
\begin{enumerate}
    \item MySQL
    \item Microsoft SQL Server
    \item Oracle
\end{enumerate}
  \par Pada era industri 4.0 ini dan pada tahun dahulu basis data sudah diterapkan,contohnya pengelompokkan data prajurit,dompet yang kelompokkan letak katu-kartu,struk belanja dll. Contoh lainnya yaitu seperti lemari yang sebagai package atau pembungkus dari isi yang ada di dalam dan kemudian isi dari lemari dipisahkan, misal baju dibagian paling atas,celana dibagian tengah,dan bagian bawah diisi dengan yang lainnya.
  
\par Database berawal dari himpunan matematika,berelasi antar himpunan saat ini, karena mempunyai fitur yang sangat lengkap jika dibandingkan dengan framework.

Tujuan Database
\begin{enumerate}
    \item kemudahan&speed
    \item akurasi
    \item keamanan
    \item ketersediaan
    \item kelengkapan
    \item space
    \item kebersamaan pemakaian

Data : Sekumpulan informasi yang berupa nilai, catatan/tulisan yang berupa angk/huruf(nilai/value yang mempresentasikan).
Base : Tempat/wadah penyimpanan data.
\end{enumerate}