\documentclass[12pt]{article}
\usepackage{amsmath}
\usepackage{graphicx}
\usepackage{hyperref}
\usepackage[latin1]{inputenc}

\title{TUGAS MATA KULIAH BASIS DATA 1}
\author{oleh Nur Ikhsani Suwandy Futri\\1194029\\D4 Teknik Informatika\\1A}
\date{Thursday/27/02/2020}

\begin{document}
\maketitle


Basis data atau sering di dikenal dengan database adalah suatu kumpulan data atau suatu arsip yang dapat berhubungan tanpa adanya redudansi yang dapat disimpan dengan cara sitematis pada komputer sehingga dapat diolah dengan perangkat lunak atau program aplikasi  sehingga dapat menghasilkan suatu informasi yang dapat di lihat oleh user. Database dalam pengguaannya sangat lah penting bagi para progremer karena database merupakan penyimpanan suatu informasi karena semakin banyak informasi yang diberikan dan diperbaharui maka akan semakin banyak ruang penyimpan yang dibutuhkan sehingga database ini sangat lah penting bagi para progremer dalam permbuatan program.

DBMS adalah sistem yang mengolah serta mengorganisasikan database pada komputer yang berupa perangkat lunakyang dapat menbuat basis data yang bersifat komputerisasi. DBMS berperan sebagai perantara dari user dengan database contohmya adalah mysql 

Tujuana yang utama dari database adalah sebagai salah satu alternatif agar dapat lebih cepat dan mudah dalam menemukan suatu arsip/data yang dulu pernah ada atau dapat pula bertujuan sebagai tempat penyimpanan berupa data untuk memudahkan bagi para progremer dalam menyampaikan informasi atau membuat suatu program. Karena pada dasarnya dalam penyampaian dan pembuatan suatu program memerlukan tempat atau wadah yang sangat banyak maka dalam peciptaanya database ini memiliki tujuan yaitu sebagai wadah yang dapat digunakan oleh semua orang khususnya para progremer. Dalam penciptaanya database memiliki manfaat untuk memudahakan seseorang dalam mencari data dari  suatu data yang tidak dapat diterhitung serta memakan waktu yang lama dan telah ada sejak dulu, seperti halnya yang ada di suatu desa maka akan ada banyak data dari tiapa kepala keluarga maka dibutuhkan suatu database yang dapat menampung serta dapat memudahkan dalam pencariannya ketika dibutuhkan data dari satu anggota keluarga yang berada di satu desa tersebut.
selain penjelasan di atas database juga memilki tujuan lain yaitu :

\begin{itemize}
  \item sebagai suatu kelengkapan. 
  \item sebagai keamanan. 
  \item sebagai keakuratan.
\end{itemize}

Perbedaan antara basis data tradisionaldan modern yaitu pada database modern terdapat banyak pembaharuan dan memiliki kemudahan yang sangat mendukung bagi sesorang atau seorang programer sedangkan pada basis data tradisional masih menggunakan data yang manual sehingga tidak terlalu effisien dalam pelaksanaanya serta masih terlalu banyak kekurangan. 

Contoh penerapan database 
Pada zaman dahulu serta masih digunakan sampai sekarang yaitu rak piring dimana pengelompokan rak piring terdiri dari kelompok piring terdapat pada tempatnya tersendiri lalu ada kelompok gelas kolompok sendok dan garpuh serta wadah wadah yang digunakan untuk tempat yang besar tersusun rapih di dalam rak piring tanpa adanya suatu tempat yang digandakan seperti contoh gelas yang adadi dalam tempat piring karena pada dasarnya telah terdapat tempatnya sendiri.
\end{document}
