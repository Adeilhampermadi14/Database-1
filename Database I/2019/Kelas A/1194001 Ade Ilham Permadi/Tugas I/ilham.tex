\documentclass[12pt]{article}
\usepackage{amsmath}
\usepackage{graphicx}
\usepackage{hyperref}
\usepackage[latin1]{inputenc}

\title{Rangkuman Basis Data}
\author{ADE ILHAM PERMADI}
\date{27 february 2020}

\usepackage[left=3.00 cm, bottom=4.00 cm, right=3.00 cm, top=2.00 cm]{geometry}

\begin{document}
\maketitle


\section{Pengertian Basis Data}
 
    
    Himpunan kelompok data yang saling terhubung dan diorganisasi sedemikian rupa supaya kelak dapat dimanfaatkan kembali secara cepat dan mudah.
    Kumpulan data dalam bentuk file/tabel/arsip yang saling berhubungan dan tersimpan dalam media penyimpanan elektronis, untuk kemudahan dalam pengaturan, pemilahan,  pengelompokan dan pengorganisasian data sesuai tujuan.
    
    
    contoh yang sederhana
  

\begin{itemize}
\item dompet
\item lemari rumah
\item berangkas kantor camat
\end{itemize}
  
\section{Database Management System atau DBMS}

DBMS atau database management system adalah program aplikasi khusus yang dirancang untuk membuat dan juga mengelola database yang tersedia. Sistem ini berisi koleksi data dan set program yang digunakan untuk mengakses database tersebut.

DBMS adalah software yang berperan dalam mengelola, menyimpan, dan mengambil data kembali. Adapun mekanisme yang digunakan sebagai pelengkap adalah pengaman data, konsistensi data dan pengguna data bersama.

berikut contohnya

\begin{itemize}
\item MySQL
\item oracle
\end{itemize}

  
\section{Tujuan basis data}
 tujuan dari basis data ini adalah :
\begin{enumerate}
\item untuk memudahkan menyimpan data,melakukan perubahan data dan menampilkan kembali data dengan lebih cepat dan mudah dibandingan cara manual

\item untuk efisien ruang penyimpanan.adanya basis data yang ini akan mengurangi kerangkapan data dan menghemat ruang

\item untuk menghindari redudansi data atau data ganda yang dapat membuat menjadi sampah di penyimpanan 
\end{enumerate}



\end{document}