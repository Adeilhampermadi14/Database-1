\documentclass{article}
\usepackage[utf8]{inputenc}

\title{DATABASE DESIGN (BASIS DATA 1)}
\date{}

\maketitle




\begin{center}
\vspace{4cm}
Oleh :
Ririn Indriyani\\
D4 TI 1B\\
1194065\\
\vspace{4cm}
\textbf{PROGRAM DIPLOMA IV TEKNIK INFORMATIKA} \linebreak
\textbf{POLITEKNIK POS INDONESIA} \linebreak
\textbf{BANDUNG} \linebreak
\textbf{2020} \\

\end{center}


\begin{document}

\maketitle
\newpage
\section{Sejarah Database}
Database berasal dan muncul dari dari suatu masalah
seseorang dalam kehidupanya.proses pembuatan data manual dan menggunakan sistem kertas atau berkas sehingga memakan ruang dan tempat yang banyak,dan apabila berkas dan kertas tersebut digunakan maka harus di cari di rak-rak atau tempat penyimpananya sehingga menghabiskan waktu dalam mencarinya.

\section{Pengertian Database}
Database berasal dari 2 kata yaitu “Basis” dan “Data”.
\vspace{0,3cm}
\par a. Basis (penyimpanan)
\par - Kumpulan
\par - Gabungan
\par - Tempat
\par - Bidang
\vspace{0,5cm}
b. Data (merupakan kebutuhan primer)
\par - Gambar
\par - Informasi
\par - Simbol
\par - Nilai
\par - Fakta
\par - Angka.
\vspace{0,3cm}
\par Database adalah sebuah system data yang memiliki objek berupa angka,huruf,dan symbol yang di kumpulkan dalam sebuah tabel data yang saling berelasi(berhubungan).
\section{Tujuan Database}
\par a. Sebagai tempat penyimpanan data,karena database             membutuhkan penyimpanan besar dan tidak mampu dengan         cara penyimpanan data manual.
\par b. Dengan menggunakan teknologi akan mempermudah               pengelompokan data berdasarkan fungsi masing-masing
\par c. Memecahkan masalah penyimpanan yang membutuhkan             ruang penyimpanan besar
\par d. Mengurangi perulangan kata.
\section{Penerapan Database}
\par Dalam proses penerapan Database harus sesuai dengan kenyataan dan terstruktur.
\par - Contoh Basis data yang menggunakan system manual atau        tradisional yaitu, Lemari,Dompet,Tas.
\par - Contoh Basis data yang menggunakan system teknologi         sekarang yaitu, ktp,ktm,kasir,networking.
\vspace{0,3cm}
\par Dalam proses pembuatan database harus mengetahui dan mengelompokan bagian dari pada normalisasi.Normalisasi ialah pengelompokan atribut data dalam membentuk entitas sederhana,serta data dalam tabelnya harus di pilih sesuai fungsi masing-masing,teratur dan saling berelasi.
Fungsi dari Database ialah untuk menghindari “Redudance”.redudance sendiri berarti perulangan data.
Dalam pembuatan data pemrograman sangat berkaitan erat dengan database.satu kodingan yang salah saja maka web ataun kodingan yang kita buat akan error.
Dalam sebuah web pemrograman, database di tempatkan pada suatu tempat tertentu yang bernama MSQL(Management Structured Query Laguage), yang berfungsi akan menampilkan data pada web pemrograman yang kita buat.
\section{Tahapan Pembuatan Database}
\par 1. Analisa 
yaitu suatu hal yang di analisa terlebih dahulu sebelum menjadi,
\par 2. Database
yaitu akan menjadi pada suatu tabel data
\par 3. Pemrograman
Yaitu melalui kedua tahap di atas.
\end{document}
