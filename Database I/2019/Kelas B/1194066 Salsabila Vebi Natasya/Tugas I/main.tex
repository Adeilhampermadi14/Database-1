\documentclass{article}
\usepackage[utf8]{inputenc}



\begin{document}
\title{\huge\textbf{Rangkuman database}}
\date{}
\maketitle



\begin{center}
\vspace{4cm}
Salsabila Vebi Natasya\\
D4 TI 1B\\
1.19.40.66\\
\vspace{4cm}
\textbf{PROGRAM DIPLOMA IV TEKNIK INFORMATIKA}\linebreak
\textbf{POLITEKNIK POS INDONESIA}\linebreak
\textbf{BANDUNG}\linebreak
\textbf{2020}

\end{center}
\newpage
\section{DATABASE}
\par Basis data adalah kumpulan data berupa angka,huruf,gambar,symbol yang diletakkan pada satu tempat yang mana datanya harus sesuai fakta dan saling berelasi.
\par Basis data berasal dari 2 kata :
\par Basis yang bisa diartikan penyimpanan dan bisa juga diartikan :
\par •	Kumpulan
\par •	Gabungan
\par •	Tempat
\par •	Bidang
\par Contoh tempat penyimpanan basis data adalah MYSQL
\par Data yang bisa diartikan :
\par •	Gambar 
\par •	Symbol
\par •	Informasi
\par •	angka
\par •	fakta
\par contoh basis data dalam kehidupan sehari-hari :
\par •	Lemari 
\par •	KTM
\par •	Dompet
\vspace{0,5}\\
\par Sejarah basis data :
Sebenarnya sudah sejak dulu orang-orang sudah menerapkan basis data, namun yang berbeda hanya caranya masih manual contohnya:
\par •	ada seorang pria yang ingin membuat KTP karena KTP lamanya terbakar. Jadi orang petugas kecamatan harus mencari data-data pria tersebut dengan melihat satu-satu berkas yang isi nya data pria tersebut dan itu membutuhkan waktu yang sangat lama.
\par •	Menyimpan baju di dalam lemari disusun sesuai dengan pakaian dalam,celana,baju baru ,baju lama yang sudah jarang digunakan. Jika kita ingin mencari baju yang sudah lama akan susah Karena terlalu banyak baju yang ada dilemari.
\vspace{0,5}\\
\par Berdasarkan hal itulah kenapa dibuat basis data menggunakan teknologi, sedangkan kelebihan menggunakan basis data teknologi :
\par 1.	Lebih hemat waktu dalam mencari data-data, karena jika  mencari data satu-satu akan membutuhkan waktu yang lebih lama.
\par 2.	Lebih aman, karena data bisa dibackup untuk menghindari  hal-hal yang tidak diinginkan 
Contohnya seperti kebakaran.
\par 3.	Tidak memakan tempat dan biaya, karena jika menggunakan  tempat penyimpanan seperti lemari untuk menyimpan data-data akan  membutuhkan lemari lagi jika lemari satunya sudah penuh.
\vspace{0,5}\\
\par Basis data dalam penerapannya:
\par •	Dalam penerapannya basis data harus terstruktur atau  diletakan sesuai type data nya.
\par •	Dalam pembuatan basis data kita harus melakukan normalisasi  untuk menghindari redudance atau data yang berulang-ulang.


\end{document}
