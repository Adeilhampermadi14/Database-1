\documentclass{article}
\usepackage[utf8]{inputenc}

\title{Basis Data}
\author{ }

\date{Senin,24 February 2020}

\begin{document}

\maketitle
Dosen Pengampu :Syafrial Fachri pane,ST,MTI,EBDP.
\newline

Nama :Rendi Baji Syadewo
\newline

NPM :1194064
\section{Explanation}
Basis Data terdiri dari kata basis dan data.
Basis dapat di artikan  sebagai Kumpulan,Gabungan,Tempat,Bidang.
Sedangkan Data adalah Catatan,Informasi,Simbol,Nilai,Fakta,
Angka.
Pengertian Basis Data atau Data Base adalah Kumpulan file/informasi 
yang saling berhubungan yang tersimpan dalam media penyimpanan elektronis/Online.
\section{Contoh Basis Data Tradisional}
Seperti Lemari,Dompet,Tas.Kita ambil contoh Lemari yg didalamnya terdapat baju,celana,Dll.
Tersusun secara Terstruktural.jika beban melebihi batas maka kita perlu membeli lemari lagi dan itu memerlukan biaya tambahan.
\section{DBMS (Data Base Management System)}
DBMS adalah pengolahan database pada komputer atau laptop system ini merupakan perangkat lunak untuk membangun basis data
\section{Redudansi dan Normalisasi}
\begin{enumerate}
    \item 
Redudansi adalah kejadian berulang-ulang data atau kumpulan data   yang sama dalam sebuah database.Contoh Nama (Rendi,Rendi,Rendi) itu  termaksud redudansi karena terjadi pengulangan kata Rendi sebanyak   3 kali.
    \item
Normalisasi adalah proses pengelompokan suatu atribut yang terdapat dalam data base  sehingga  database yang dibuat berkualitas dengan baik

\end{enumerate}
\end{document}
