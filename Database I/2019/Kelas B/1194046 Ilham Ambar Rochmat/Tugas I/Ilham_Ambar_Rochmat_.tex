\documentclass{article}
\usepackage[utf8]{inputenc}

\title{Rangkuman  Basis Data}
\author{Ilham Ambar Rochmat / 1194046 }
\date{Senin, 24 feb 2020}

\begin{document}

\maketitle
\section{Basis Data}
\begin{document}

\maketitle

Basis Data yaitu tempat untuk mengumpukan atau penyimpanan, type-type data yang di satukan menjadi satu. dan untuk menghindari normalisasi dan redudansi.

\section{Sejarah Dan Manfaat Basis Data}

\maketitle

 Terbentuknya basis data dari tradisional hingga digital
 contoh, basis data tradisional yaitu : Lemari, dopet, tas, rak diantara itu memiliki funggi masing-masing agar tersusun dengan rapih. Manfaatnya yaitu untuk mempermudah pengguna mencari data, efisien, terhindar data dari bencana.

\end{document}


\end{document}
