\documentclass{article}
\usepackage[utf8]{inputenc}

\title{Basis Data}
\author{Ardhila Pudjaningrum}
\date{February 2020}

\begin{document}

\maketitle

\section{Pengertian Basis Data}
Basis data adalah kumpulan data berupa angka, huruf, gambar, symbol yang dikumpulkan didalam satu tempat dimana datanya tersusun berdasarkan fakta dan saling berelasi.

\section{Tujuan Basis Data}
1. Mempermudahkan identitas data, database menyiapkan data sesuai permintaan user terhadap suatu informasi dengan cepat dan tepat.

2. Menghindari data ganda

3. Mempermudah dalam mengakses, penyimpanan, mengedit, dan menghapus data.

4. Menjaga kualitas data dan informasi agar tetap sama.

5. Mendukung aplikasi yang membutuhkan ruang penyimpanan.

\section{DBMS (DataBase Management System)}
Database Management System (DBMS) adalah software atau perangkat lunak untuk membuat database dalam sebuah komputer.

\end{document}