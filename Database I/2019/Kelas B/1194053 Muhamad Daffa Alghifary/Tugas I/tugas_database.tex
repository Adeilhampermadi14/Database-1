\documentclass{article}
\usepackage[utf8]{inputenc}

\title{Basis Data}
\author{m.daffa.algifary}
\date{February 2020}

\begin{document}

\maketitle

\section{Pengertian }
Data base adalah kumpulan data Dan informasi yang di simpan yang di dalam computer secara sitematis dan terperinci sehinga dapat di periksa di gunakan dalam satu komponen yaitu computer di dalam dan basis data berawal dari ilmu komputer dan arti itu semakin luas,karena basis data dapat di harfiahkan dalam kehidupan sehari hari dan dapat di mengerti dengan sangat mudah dan kita suka temui dalam keseharian contohnya dalam mengunakan data-data menuntukan kartu tanda mahasiswa sebagai wadah 


\section{Tujuan basis data}
Tujuan basis data adalah di ciptakan untuk memudahkan kita untuk melakukan pekerjan sehari hari yang bergantung dengan koputer supaya mempermudah kita

1. pencarian sebuah data yang akan banyak memakan waktu secara manual oleh karena itu, diperlukan sebuah teknologi yang dapat mencari data secara efektif dan efisien
2. mengurutkan data yang dilakukan secara manual juga akan memakan waktu, oleh sebab itu teknologi ini diciptakan.

3.contoh data base dalam sehari hari lemari kulkas tas dompet dan ktm

4.jika mengunakan data yang di kirim secara manual sangat tidak efisien dan sangat memakan waktu yang sangat lama jika sekarang masih mengunakan secara manual waktu yang kita kerjakan sabgat lah lama



\section {inti dari data base}
mempermudah kita mengunakan mengirim data dan kita tidak perlu jalan jika kita jaman dahulu kita mengunakan data sangat lama harus mengirim pesan harus kita juga akan meghabis kan waktu dantenaga dan pula termasuk uang 

\section {data base managemen }
data yang dapat di simpan  dbms buat server 
seperti MYSQL

\end{document}