\documentclass{article}
\usepackage[utf8]{inputenc}

\title{Rangkuman Basis Data 1}
\author{1194061 - Nur Raisa D4 TI 1B}
\date

\begin{document}

\maketitle

\section{Basis Data}
Basis : (tempat penyimpanan).
\begin{enumerate}
    \item Kumpulan
    \item Gabungan
    \item Tempat
    \item Bidang
\end{enumerate}
Data :
\begin{enumerate}
    \item   Nilai
    \item   Gambar
    \item   Informasi
    \item   Sysmbol
    \item   Fakta
    \item   Waktu
\end{enumerate}
Basis Data : Kumpulan-Kumpulan yang telah dinormalisasikan untuk menguranggi redudansi (terstruktur) yang disimpan di dalam komputer secara sistematik sehingga dapat diperiksa menggunakan suatu program komputer untuk memperoleh informasi dari basis data tersebut. 
Redudansi adalah pengulangan kata atau data yang sama tidak dapat dilacak tapi dapat di cek.
Aplikasi basis data : MySQL dan ORACLE.
Software atau DBMS (Data Base Management Sistem).
Contoh basis data dalam kehidupan sehari-hari :
\begin{enumerate}
    \item Lemari
    \item Dompet
    \item Tas
    \item Kulkas
    \item Gudang
\end{enumerate}
Alasan terciptanya Basis Data :
\begin{enumerate}
    \item Kurangnya ruang penyimpanan.
    \item Menghemat budget
    \item Memudahkan mencari data
    \item Adanya backup datanya
\end{enumerate}
\end{document}