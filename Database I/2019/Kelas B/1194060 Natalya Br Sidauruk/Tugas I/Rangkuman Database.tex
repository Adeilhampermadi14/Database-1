\documentclass{article}
\usepackage[utf8]{inputenc}

\title{1194060 Natalya Br Sidauruk D4 TI 1B}
\author{Tugas Rangkuman Database }
\date{24 February 2020}

\usepackage{natbib}
\usepackage{graphicx}

\begin{document}

\maketitle

\section{ Pengertian }

\indent Basis data adalah kumpulan kumpulan yang sebuah data yang dapat berupa angka, huruf, gambar, symbol yang disatukan di satu tempat yang datanya fakta saling berelasi dan  telah normalisasi melawati redudansi. Redunsi gunanya untuk mengecek data agar tidak terjadi data yang sama. 

\indent Pada computer mengambil dan memasukkan data menggunakan Database Management System (DBMS). Contoh aplikasi Database adalah Oracle dan MySQL. Dalam Database memiliki tabel dan jumlah tabelnya lebih baik satu karena jika kita meletakkan dalam satu tabel maka isi tabel tidak akan sistematis, sehingga kita harus menambahkan tabel pada database dan setiap tabel dibuat saling terhubung.


\section{ Tujuan }
Tujuan dari Basis Data :
\begin{enumerate}
    \item Jika terjadi sebuah insiden data tidak dapat hilang
    \item Menghemat budget
    \item Lebih mudah mencari suatu data/nama data dalam setiap data
    \item Agar data dapat up-to-date terus menerus
    \item Agar tidak terjadinya duplikasi  data
    \item Efesiensi dalam penyimpanan
    \item Dapat dibuka dimana saja dengan terhubung oleh MySWQL atu Oracle
\end{enumerate}

\section { Contoh }
Contoh yang termaksud dalam database dalam kehidupan :
\begin{enumerate}
    \item Lemari dan Kulkas    
    \item Dompet dan Gudang
\end{enumerate}
\end{document}
