\documentclass{article}
\usepackage[utf8]{inputenc}

\begin{document}
\title{\huge\textbf{BASIS DATA}}
\date{}


\maketitle


\begin{center}
\vspace{4cm}
Oktaviami Manullang\\
D4 TI 1B\\
1.19.40.62\\
\vspace{4CM}
\textbf{PROGRAM DIPLOMA IV TEKNIK INFORMATIKA} \linebreak
\textbf{POLITEKNIK POS INDONESIA} \linebreak
\textbf{BANDUNG} \linebreak
\textbf{2020}\\

\end{center}
\newpage
\section{Basis Data}
Basis : Tempat Penyimpanan
\par - Gabungan
\par - Kumpulan
\par - Tempat
\par - Bidang
\vspace{0,5cm}
\par Data : Sekumpulan informasi nilai, angka, huruf yang harus falid dan konsisten
\par -	Gambar
\par -	Informasi
\par -	Nilai
\par -	Simbol
\par -	Angka
\vspace{0,5cm}
\par Basis Data merupakan kumpulan informasi yang dikelompokkan berupa, gambar, simbol yang dikumpulkan pada satu tempat yang mana datanya harus saling berelasi.
\vspace{0,5cm}
\par Tujuan Basis Data :
\par •	Sebagai tempat penyimpanan data
\par •	Menentukan kualitas informasi yang akurat dan relevan
\par •	Menghindari data ganda dan inkonsistensi data
\par •	Memecahkan masalah penyimpanan konvensional
\vspace{0,5cm}
\par Contoh Basis Data Tradisional :  
\par -	Lemari
\par -	Tas
\par -	Dompet
\par -	Kulkas
\newpage
\par Contoh Basis Data Teknologi :
\par -	KTM
\par -	Jaringan
\par -	Kasir
\vspace{0,5cm}
\par Dalam pembuatan Basis data harus mengetahui normalisasi dan redudansi.
\par •	Normalisasi merupakan pengelompokkan atribut data yang membentuk entitas yang sederhana, serta data harus dipisah sesuai fungsinya masing-masing 
\par •	Tujuan Normalisasi Basis Data ialah untuk menghilangkan dan mengurangi redudansi data
\par •	 Redudansi merupakan perulangan kata

\section { Sejarah Basis Data}
     Pada awalnya basis data ditemukan di kehidupan kita sehari-hari, dari sinilah awal mulanya dibentuk sistem basis data. Basis data terbentuk pada suatu kebiasaan manusia secara manual namum membutuhkan tempat dan ruang yang banyak
\vspace{0,5cm}
\par Contoh manual :
Mengambil atau mencari data-data masyarakat secara manual (dicari satu-persatu di berkas yangberupa kertas).



\end{document}
