\documentclass{article}
\usepackage[utf8]{inputenc}
\title{Basis Data}
\author{}
\date{24 Februari 2020}

\begin{document}

\maketitle
Dosen Pengampu : Syafrial Fachri Pane, ST., MTI., EBDP.
\newline

Nama : Muhammad Nanda Fahriza
\newline

Npm : 1194057
\section{Pengertian Basis Data}


Basis Data terdiri dari dua kata yaitu basis dan data, basis yang artinya kumpulan atau gabungan, data adalah nilai bisa terdiri dari angka, informasi, gambar,dan simbol.



Basis data adalah gabungan dari data - data yang di simpan di media tekhnologi sehingga dapat diakses dan dicari.Database juga ada dalam kehidupan kita sehari-hari contoh nya seperti lemari,tas,dompet dll, contoh sebelumnya adalah sebagai basis dan data terdiri dari pakaian, buku, uang,dll.

\section{Manfaat Basis Data}
Banyak sekali manfaat dari menggunakan database bisa mempermudah untuk mencari data, menghapus data, menginput data, dan terjamin keamanan datanya, keuntungan menggunakan database berbasis teknlogi dengan database manual adalah jika sistem dari database nya rusak maka data yang ada didalamnya dapat di back-up maka datanya akan bisa ada kembali,berbeda dengan menggunakan database berbasis manual yang datanya tidak dapat back-up apabila terjadi kerusakan atau kehilangan dari datanya.



\section{Normalisasi dan Redudance}
Normalisasi adalah pengelompokkan kelengkapan data yang membentuk entitas sederhana, tidak ada data yang sama, data yang berulangan sehingga bisa dipastikan database yang sudah dibuat berkualitas baik.



Redudance adalah data yang berlebihan atau duplikasi data, perulangan data yang disimpan dalam beberapa file.



\section{Database Management System}
Database Management System atau kepanjangan dari DBMS adalah sistem yang mengolah atau penorganasasian database pada komputer.



Dibawah ini beberapa contoh dari DBMS :
\begin{enumerate}
    \item 
     MySQL
     \item
     Oracle
     \item
     MicrosoftSQLserver
    

\end{enumerate}

    




 
\end{document}
