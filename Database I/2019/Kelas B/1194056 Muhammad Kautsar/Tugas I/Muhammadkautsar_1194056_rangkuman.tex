\documentclass[a4paper,12 pt]{article}
\usepackage{color}
\usepackage[bahasa]{babel}
\usepackage{graphicx}
\title{\textbf{Rangkuman}\linebreak \\ \textbf{Basis Data}\linebreak}
\date{}
\begin{document}
\maketitle
\textbf {Basis Data} \linebreak
\end{center}
\vspace{0.5 cm}
\begin{center}
\begin{tabular}{11}
Nama & : Muhammad Kautsar\\
NPM & : 1194056\\
Kelas & : D4 TI 1B\\
\end{tabular}
\newline
\newline
\newline
Untuk Memenuhi Tugas Basis Data \\
Dosen Pengampu: Syafrial Fachri Pane, ST., MTI., EBDP. \linebreak
\newline

\newline
Program Study of Informatics Engineering \\
\textit {Politeknik Pos Indonesia}
\linebreak
Bandung 2020 \linebreak
\end{center}
\newpage
	Basis data adalah kumpulan dari data ataupun informasi yang dikelompokan dalam suatu  tempat secara beraturan agar mudah dalam pencariannya. Contoh pemanfaatan basis data dalam kehidupan sehari hari seperti lemari, gudang, tas, dan lain-lain. 
	Implementasi basis data pada komputer dilakukan menggunakan software khusus database seperti Microsoft acces, Oracle, Ms SQL Server, MySQL, dan lain-lain. Dalam proses pengolahannya, database harus melawati proses normalisasi agar menghindari atau mengurangi kemungkinan redudansi (data yang sama) pada data tersebut. 
	Dalam pembuatan web ataupun apalikasi, nama database harus jelas dan unik agar tidak terjadi error pada programnya ataupun kesalahan data. Selain itu ada beberapa istilah yang pada database seperti DBMS (yaitu sebuah sotware yang memberikan akses bagi penggunanya untuk membuat, merubah, atau menghapus database), Dan RDBMS (yaitu sebuah tools untuk menghubungkan antarbasis data). 
Manfaat dari database sendiri yaitu mempermudah dalam melakukan pencarian, dan lebih efisien dalam waktu.
\end{document}
