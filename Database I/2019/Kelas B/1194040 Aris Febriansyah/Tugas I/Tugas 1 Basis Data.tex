\documentclass{article}
\usepackage[utf8]{inputenc}

\title{Basis Data}
\author{aris.febriansyah2001 }
\date{February 2020}

\begin{document}

\maketitle

\section{Pemahaman Basis Data}
Basis data merupakan pengklompokan  data yang saling terorganisir anatara yang satu dengan yang lainnya dengan maksud supaya kelak dapat di manfaatkan secara cepat dan mudah 

\section{Fungsi Basis Data}
Fungsi dari Basis Data atau Database mencakup lingkungan yang begitu luas dalam mendukung keberadaan lembaga atau organisasi
Dan inilah beberapa fungsi atau tujuan dari Basis Data

 A.Kemudahan 
 
 Dengan Basis Data maka kita akan dengan gampang dan cepat untuk melakukan suatu kerjaan yang kita inginkan selain memudahkan kerjaan kita dalam kehidupan kita, Basis Data juga membantu pekerjaan kita dengan cepat sehingga mengeefisiensikan waktu kita lebih cepat dan lebih berguna
 
 B.Storage Efficiency
 
 Data yang telah terorganisir dengan baik akan menghindari dari duplikasi data dengan demkikian kita bisa menaggulangi resiko pengurangan ruang penyimpanan Basis Data yang di akibatkan data yang sama 
 
 C.Accuracy
 
 Accuracy atau keakuratan merupakan salah satu fungsi berikutnya dimana kita akan di bantu dengan keakuratan yang sangat akurat dengan demikian kita dapat menekan angka kesalahan yang mungkin dapat terjadi akibat kesalahan kita atau sistem.
 
 dari tiga fungsi atau tujuan dari basis data di atas merupakan beberapa di antaranya 
 
 \section{Database Management System}
 Database Management System atau DBMS adalah pengklompokan dan pengolahan database pada komputer dan biasanya DBMS merupakan perangkat lunak yang di pakai untuk membuat basis data yang berbasis komputer 
\end{document}
