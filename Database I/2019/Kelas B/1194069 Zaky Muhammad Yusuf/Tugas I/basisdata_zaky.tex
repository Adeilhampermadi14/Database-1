
\documentclass[a4paper,12 pt]{article}
\usepackage{color}
\usepackage[bahasa]{babel}
\usepackage{graphicx}
\title{\textbf{Rangkuman}\linebreak \\ \textbf{Basis Data}\linebreak}
\date{}
\begin{document}
\maketitle
\begin{center}
\includegraphics[width=5cm,height=5cm]{politeknik pos indonesia.png}
\end{center}
\begin{center}
\textbf {Basis Data} \linebreak
\end{center}
\vspace{0.5 cm}
\begin{center}
\begin{tabular}{11}
Nama & : Zaky Muhammad Yusuf \\
NPM & : 1194069\\
Kelas & : D4 TI 1B\\
\end{tabular}
\newline
\newline
\newline
Untuk Memenuhi Tugas Basis Data \\
Dosen Pengampu: Syafrial Fachri Pane, ST., MTI., EBDP. \linebreak
\newline

\newline
Program Study of Informatics Engineering \\
\textit {Politeknik Pos Indonesia}
\linebreak
Bandung 2020 \linebreak
\end{center}
\newpage
\section{}
Basis data adalah kumpulan sebuah informasi yg di jadikan satu, yg mempunyai banyak ruang penyimpanan.
Penemu basis data adalah edgar codd di laboraturium san jose.
Basis data biasa menggunakan msql. Basis data adalah suatu penghubung Normalisasi dan redudansi agar tidak terjadi nya data yg sama dan melakukan perulangan.
Keuntungan menggunakan basis data “msql” adalah sbagai berikut:
,	Lebih aman
,	Data yg tersusun secara rapih
,	Lebih cepat
,	Lebih up to date
,	Tingkat presisinya tinggi
,	Lebih mudah kemandirian data

\end{document}