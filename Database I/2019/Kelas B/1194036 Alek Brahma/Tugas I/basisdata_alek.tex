\documentclass[a4paper,12 pt]{article}
\usepackage{color}
\usepackage[bahasa]{babel}
\usepackage{graphicx}
\title{\textbf{Rangkuman}\linebreak \\ \textbf{Basis Data}\linebreak}
\date{}
\begin{document}
\maketitle
\begin{center}
\textbf {Basis Data} \linebreak
\end{center}
\vspace{0.5 cm}
\begin{center}
\begin{tabular}{11}
Nama & : Alek Brahma\\
NPM & : 1194036\\
Kelas & : D4 TI 1B\\
\end{tabular}
\newline
\newline
\newline
Untuk Memenuhi Tugas Basis Data \\
Dosen Pengampu: Syafrial Fachri Pane, ST., MTI., EBDP. \linebreak
\newline

\newline
Program Study of Informatics Engineering \\
\textit {Politeknik Pos Indonesia}
\linebreak
Bandung 2020 \linebreak
\end{center}
\newpage
\section{}
Pengertian basis data dan sistem basis data – Basis data adalah, basis data terdiri dari 2 (dua) kata, yaitu kata Basis dan Data. Basis bisa di artikan sebagai markas ataupun gudang, tempat berkumpul. Sedangkan data yaitu kumpulan fakta dunia nyata yang mewakili suatu objek, seperti manusia, barang, dan lain-lain yang direkam ke dalam bentuk angka, bentuk huruf, simbol, teks, bunyi, gambar atau juga  kombinasinya.

A. Penjelasan Basis Data

Jadi arti dari basis data adalah kumpulan terorganisasi dari data – data yang saling berhubungan sedemikian rupa sehingga dapat mudah disimpan, dimanipulasi, serta dipanggil oleh penggunanya. Definisi Basis data juga dapat diartikan sebagai kumpulan data yang terdiri dari satu atau lebih tabel yang terintegrasi satu sama lain, dimana setiap user diberi wewenang untuk dapat mengakses ( seperti mengubah,menghapus dll.) data dalam tabel-tabel tersebut.

B. Tujuan Basis data sendiri adalah sebagai berikut

Kecepatan serta kemudahan dalam menyimpan, memanipulasi atau juga menampilkan kembali data tersebut.
Efisiensinya ruang penyimpanan, karena dengan basis data, redudansi data akan bisa dihindari.
Keakuratan (Accuracy) data.
Ketersediaan (Availability) data.
Kelengkapan (Completeness) data, Bisa melakukan perubahan struktur dalam basis data, baik dalam penambahan objek baru (tabel) atau dengan penambahan field-field baru pada table.
Keamanan (Security) data, dapat menentukan pemakai yang boleh menggunakan basis data beserta objek-objek yang ada didalamnya serta menentukan jenis -jenis operasi apa saja yang boleh dilakukannya.
Kebersamaan Pemakai (Sharability), Pemakai basis data bisa lebih dari satu orang, tetapi tetap menjaga atau menghindari masalah baru seperti: inkonsistensi data (karana data yang sama diubah oleh banyak pemakai pada saat yang bersamaan) dan juga kondisi deadlock (karena ada banyak pemakai yang saling menunggu untuk menggunakan data tersebut.

\end{document}

