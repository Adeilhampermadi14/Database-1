\chapter{Analisis Data Fisik}

\section{Membuat Data Universal}
Dalam pembuatan data universal ini kita menuliskan semua atribut universal yang dimiliki oleh setiap berkas, sebagai berikut :
\begin{itemize}

	\item Kartu Keluarga\\
		Nomor Induk Kartu Keluarga\\
		Nomor Kartu Keluarga\\
		Nama Kepala Keluarga\\
		Alamat\\
		RT/RW\\
		Desa/Kelurahan\\
		Kecamatan\\
		Kabupaten/Kota\\
		Kode Pos\\
		Provinsi\\
		Nama Lengkap\\
		NIK\\
		Jenis Kelamin\\
		Tempat Lahir\\
		Tanggal Lahir\\
		Agama\\
		Pendidikan\\
		Jenis Pekerjaan\\
		Status Perkawinan\\
		Hubungan Keluarga\\
		Kewarganegaraan\\
		Nomor Passport\\
		Nomor KITAS/KITAP\\
		Nama Ayah\\
		Nama Ibu\\
		Jumlah Anggota\\
		Tanggal, Bulan, dan Tahun terbit KK\\
		Tanda Tangan\\
		Nomor Induk Pejabat\\
		Nama Pejabat

	\item Akta Kelahiran\\
		Nomor Akta\\
		Kode CSL\\
		Dari Daftar\\
		Nomor STBLD\\
		Tanggal Kelahiran\\
		Bulan Kelahiran\\
		Tahun Kelahiran\\
		Jenis Kelamin\\
		Nama Kelahiran\\
		Urutan Anak\\
		Tanggal Bulan Tahun Surat Keluar\\
		Nama Pejabat\\
		Tanda Tangan Pejabat\\
		Nama Ayah\\
		Nama Ibu\\
		Kota dan Hari Kelahiran
		
	\item Formulir F-1.21\\
		Pemerintah Provinsi\\
		Pemerintah Kabupaten/Kota\\
		Kecamatan\\
		Kelurahan/Desa\\
		Jenis Permohonan KTP\\
		Nama Lengkap\\
		No. KK\\
		NIK\\
		Alamat\\
		RT\\
		RW\\
		Kode Pos\\
		Pas Foto 2x3\\
		Cap Jempol\\
		Tanda Tangan\\
		Kota\\
		Tanggal, Bulan, Tahun Pengisian Formulir\\
		NIP\\
		Nama Pejabat

	\item Akta Cerai\\
		Nomor Akta\\
		Tanggal Cerai\\
		Nama Suami\\
		Nama Istri\\
		Umur\\
		Agama\\
		Pekerjaan\\
		Alamat\\
		Perceraian ke\\
		Penggugat\\
		Tanggal, Bulan, Tahun  dibuatnya akta cerai\\
		Nama Pejabat\\
		Tanda Tangan\\
\end{itemize}