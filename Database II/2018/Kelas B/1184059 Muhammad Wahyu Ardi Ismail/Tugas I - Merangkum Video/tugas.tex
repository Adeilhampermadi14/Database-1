\documentclass[a4paper,12pt]{report}
\begin{document}
\chapter{PENDAHULUAN}

\section{Oracle Apex}
\subsection{Apa Itu Oracle Apex}

\paragraph{}
	Oracle Apex adalah suatu platform pengembangan kode rendah yang memungkinkan anda membangun sebuah aplikasi perusahaan yang dapat diskalakan dan aman dengan fitur world-class yang dapat digunakan perangkat lunak apapun. Oracle apex juga merupakan lingkungan pengembangan perangkat
lunak berbasis web yang berjalan pada database.


\section{Spreedsheet}
\par Spreadsheet digunakan untuk menyimpan suatu informasi. Pada setiap kolomnya bisa menyimpan berbagai informasi yang berbeda.
\\
Mengelola Data dalam Spreadsheet 
\begin{enumerate}
\item Validasi Data - manual dan rawan kesalahan
\item Integritas Data - tidak dapat menjamin keakuratan data dalam lingkun-
gan multi-pengguna
\item keamanan data - penguncian sel tidak efektif
\item Berbagi data - Excel lamban dan sulit untuk dibagikan
\end{enumerate}

\section{Jenis-jenis aplikasi yang cocok untuk Apex oracle}
\begin{enumerate}
\item Large mission-critical apps for thousand of users
\item Fill in gaps in corporate systems
\item Steamline outed business processes
\item Modernization of legacy systems
\item Self-service apps for all employees
\item Customer/Partner-facing portals
\item proof of concepts
\item Quick =-win apps (lifespan < a few months
\item Replacing spreadsheets
\end{enumerate}

\section{Oracle APEX: Distinguishing Characteristics}
\par Development IDE adalah browser web yang tidak diperlukan perangkat lunak klien,aplikasi dan disimpan dalam database sebagai data meta. Deklaratif tidak ada pembuatan kode,halaman efisien dengan hanya satu request dan satu respons. Pemrosesan data dilakukan dalam Database.

\section{Manfaat Oracle Apex Untuk Siswa}
\begin{enumerate}
\item Mempelajari SQL dan Database Relasional
\item Menguasai SQL
\item Belajar pengembangan aplikasi
\item Proyek Akademik
\item SQL dan Database Relasional
\item Praktek di laboratorium dengan SQL
\item Pengembangan aplikasi
\item Aplikasi kode rendah Dev
\item Visualisasi Data
\item Memulai secara gratis di https://apex.oracle.com Atau gunakan Oracle
Academy APEX instance
\item Menyiapkan akun siswa
\item Pelajaran, dan Panduan praktikum
\end{enumerate}

\end{document}