\clearpage
\setcounter{page}{1}
\part{Pengertian dan Cara Membuat Workspace}
\paragraph{}
Pengertian oracle apex, dan Cara membuat workspace pada oracle apex.
\section{Pengertian}
\paragraph{}
Oracle Application Express (APEX) adalah platform yang berfungsi untuk membangun aplikasi yang aman, dengan fitur kelas dunia dan dapat digunakan dimana saja.Apex memiliki tiga alat utama yang digunakan di dalamnya, yakni Application Builder, SQL Qorkshop dan Utility. Berikut ini merupakan penjelasan dari ketiga alat tersebut, di antaranya yakni :
\begin{enumerate}
    \item Application Builder
    \paragraph{} Application Builder merupakan alat yang digunakan untuk mempermudah dalam membuat, melihat, mengimport aplikasi, mengatur user aplikasi, mengatur service dan memantau aktifitas yang di lakukan pengguna.
    \item SQL Workshop
    \paragraph{} SQL Workshop ialah alat yang digunakan untuk membuat tabel dan komponennya dengan menggunakan kode PL-SQL secara manual ataupun otomatis. Selain itu, SQL Workshop juga digunakan untuk melihat struktur tabel beserta komponennya, mengimport dan mengekspor script.
    \item Utility
    \paragraph{} Utility merupakan alat yang digunakan untuk melihat report atau laporan table dan komponennya serta history pada aplikasi.
\end{enumerate}
\section{Cara membuat Workspace}
Berikut ini merupakan ururtan tata cara dalam pembuatan workspace di oracle APEX, di antaranya yaitu :\\ 
\begin{enumerate}
    \item Install APEX pada PC Anda terlebih dahulu.
    \item Kemudian buka link berikut ini http://127.0.0.1:8080/apex/f?p=4550:10:10960410872195:::::
    \item Login dengan akun yang sudah kalian buat waktu install APEX tersebut.
    \item Setalah login, maka user akan diarahkan ke halaman utama.
    \item Setelah itu klik \textit{Create Workspace}
    \item Kemudian buatlah nama dari workspace yang akan dibuat, sisanya boleh dikosongkan dan itu tidak akan berpengaruh.
    \item Lalu, klik next
    \item Langkah selanjutnya, setting pengaturannya.
    \item Ubah skemanya menjadi skema HR dengan cara klik di samping ADMIN kemudian nanti akan ada pop up yang berisi HR.
    \item Lalu klik next.
    \item Setelah itu akan muncul tampilan administrator.
    \item Atur nama username dan password untuk login nantinya.
    \item Setelah itu, atur email dengan email. 
    \item Lalu Klik next
    \item Setelah next, akan ada konfirmasi kembali mengenai workspace yang telah dibuat sebelumnya.
    \item Klik create workspace
    \item Jika telah berhasil, maka selanjutnya lanjutkan ke http://127.0.0.1:8080/apex/f?p=4550:1:12275980854321:::::
    \item Login sesuai dengan nama workspace dan username password tadi.
    \item Setelah login, maka Anda akan diminta untuk mengubah password.
    \item Jika sudah selesai tahap sebelumnya, maka akan diarahkan hal workspace
\end{enumerate}
\end{document}
