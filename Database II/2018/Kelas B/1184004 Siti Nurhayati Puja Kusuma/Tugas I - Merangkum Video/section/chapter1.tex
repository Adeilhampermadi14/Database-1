\chapter*{Resume}

\begin{enumerate}
	\item Oracle Apex adalah sarana yang digunakan untuk membuat aplikasi menggunakan database oracle itu 	sendiri. Hanya menggunakan broser web pengguna  dapat mengembangkan dan menggunakan aplikasi 				profesuonal yang cepat dan aman untuk perangkat apapun. Fitur yang didapat:
	\begin{itemize}
	Drag Dan Letakkan File Xls, Csv, Xml, Atau Json
	\end{itemize}
	
	\begin{itemize}
	\item
	 Membuat Tabel Dalam Database Otonom
	\end{itemize}
	
	\begin{itemize}
	\item Unggah Data Ke Tabel Baru
	\end{itemize}
	
	\begin{itemize}
	\item Buat Aplikasi Berdasarkan Tabel Baru
	\end{itemize}
	
	\begin{itemize}  
	\item Meluas Erps Dan Perangkat Lunak Perusahaan Lainnya.
	\end{itemize}

	\begin{itemize}
	\item Menyediakan Dasbor Khusus Organisasi.
	\end{itemize}
	
	\begin{itemize}
	\item Alur Kerja Yang Lebih Baik.
	\end{itemize}

	\begin{itemize}
	\item Mengisi Kekosongan.
	\end{itemize}
		
	\begin{itemize}
	\item Fitur Tanpa Biaya Dari Database Oracle
	\end{itemize}
	
	
	
	\item Cara membuat Workspace di Apex:
	
	\begin{itemize}
	\item Install APEX di laptop/komputer masing-masing 
	\end{itemize}
	
	\begin{itemize}
	\item
	Lalu kunjungi http://127.0.0.1:8080/apex/f?p=4050:3:12501047145942::::: di browser masing-masing
	\end{itemize}
	
	\begin{itemize}
	\item Login menggunakan akun yang sudah dibuat saat melakukan instalasi APEX sebelumnya 
	\end{itemize}
	
	\begin{itemize}
	\item Lalu akan diarahkan masuk ke halaman utama 
	\end{itemize}
	
	\begin{itemize}  
	\item Terdapat tulisan “Crate Workspace” tuliskan nama yang akan digunakan pada workspace anda masing-masing.
	\end{itemize}

	\begin{itemize}
	\item Untuk melakukan perubahan pada skemanya menjadi HR, maka klik bagian ADMIN dan akan tertera pop up berisi HR lalu klik next 
	\end{itemize}
	
	\begin{itemize}
	\item Atur ulang username dan pasword untuk melakukan login 
	\end{itemize}

	\begin{itemize}
	\item Atur ulang email dan klik next 
	\end{itemize}
		
	\begin{itemize}
	\item Lalu akan muncul tampilan konfirmasi pembbuatan workspace 
	\end{itemize}
	
	\begin{itemize}
	\item Lalu login kembali dan terdapat himbauan untuk perubahan pasword lalu akan diarahkan menuju workspace
	\end{itemize} 	
	
	\item Spreadsheet adalah memudahkan para pengguna untuk menyimpan berbagai informasi yang sangat lengkap, di setiap kolomnya dapat menyimpan berbagai data informasi yang berbeda dari informasi yang diperlukan. App form Spreadsheet berbentuk project dan nama tugas yang mempunyai keterangan tanggal mulai, tanggal selesai, status, tanda tangan, biaya dan budget.  Cara membuat aplikasi dari spreadsheet:
	\begin{itemize}
	\item Buka http://Apex.Oracle.com
	\end{itemize}
	 
	\begin{itemize}
	\item Klik get started for free
	\end{itemize}
		 
	\begin{itemize}
	\item Lalu klik permintaan ruang kerja yang kosong
	\end{itemize}
	 
	\begin{itemize}
	\item Masuk keruang kerja
	\end{itemize}
	 
	\begin{itemize}
	\item Klik pembuatan aplikasi
	\end{itemize} 
	
	\item Cara mengubah spreadsheet menjadi aplikasi di web:
	\begin{itemize} 
	\item Buka http://Apex.Oracle.com dan lakukan sign in
	\end{itemize}
	
	\begin{itemize}
	\item Membuat workspace
	\end{itemize}
	 
	\begin{itemize}
	\item Buka app builder
	\end{itemize}
	 
	\begin{itemize}
	\item Klik create new app
	\end{itemize}
	
	\begin{itemize}
	\item Lalu lakukan drag and drop file
	\end{itemize}
	 
	\begin{itemize}
	\item Lalu pilih movie, next
	\end{itemize}
	 
	\begin{itemize}
	\item Isikan pada table nama, dan jenis huruf yang akan digunakan
	\end{itemize}
	 
	\begin{itemize}
	\item Lalu klik create application
	\end{itemize}
	 
	\begin{itemize}
	\item Masuk ke appreance yang berisikan jenis tabel yang ada
	\end{itemize}
	
	\begin{itemize}
	\item Pilih icon yang diinginkan
	\end{itemize}
	 
	\begin{itemize}
	\item Di tab feature ceklis semua kolom
	\end{itemize}
	 
	\begin{itemize} 
	\item Run aplikasi dengan memasukan username dan password.
	\end{itemize}

	
	
	
	
\end{enumerate}
