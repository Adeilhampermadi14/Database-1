\documentclass{article}
\usepackage[utf8]{inputenc}

\title{Rangkuman database}
\author{Muhammad Sharif Miftahuddin(1184044) }
\date{October 2019}

\begin{document}


\maketitle

\section{Oracle Apex}
Oracle Apex merupakan platform pengembangan kode rendah yang membangun aplikasi perusahaan. Selain itu oracle apex juga merupakan lingkungan pengembangan perangkat lunak berbasis web yang berjalan pada database.

\subsection{Cara membuat workspace di apex}
berikut merupakan cara membuat workspace di apex.
\begin{enumerate}
    \item install apex
    \item buka http://127.0.0.1:8080/apex/f?p=4550:10:10960410872195:::::
    \item login dengan akun apex yang sudah dibuat saat penginstallan apex
    \item jika sudah, maka akan masuk ke halaman utama
    \item setelah itu klik create workspace
    \item buat nama dari workspacenya
    \item klik tombol next
    \item untuk mengubah skemanya menjadi HR, maka klik pop up berisi HR
    \item setelah itu klik tombol next
    \item atur username dan password untuk login
    \item atur email
    \item lalu klik tombol next
    \item setelah mengklik tombol next akan ada konfirmasi kembali menganai workspace
    \item jika sudah berhasil, maka selanjutnya pergi ke http://127.0.0.1:8080/apex/f?p=4550:1:12275
    \item maka login sesuai dengan nama workspace dan username password tadi
    \item setelah login, maka akan ada perintah untuk mengubah password
    \item jika sudah selesai, maka akan langsung menuju ke halaman workspace
    
\section{Spreadsheet}
Spreadsheet adalah dokumen yang menyimpan data dalam grid baris (rows) horisontal dan kolom (columns) vertikal. Spreadsheet digunakan untuk menyimpan berbagai informasi yang berbeda.
\subsection{Membuat aplikasi dari spreadsheet}
berikut merupakan langkah langkah untuk membuat aplikasi menggunakan spreadsheet
\begin{enumerate}
    \item pergi ke http://apex.oracle.com
    \item klik get started for free
    \item klik permintaan ruang kerja yang kosong
    \item masuk ke ruang kerja di http://apex.oracle.com
    \item klik pembuat aplikasi 6. lalu klik buat aplikasi baru

\end{enumerate}

\end{enumerate}
\subsection{Mengubah spreadsheet menjadi aplikasi web}
Berikut merupakan salah satu cara mengubah spreadshhet menjadi aplikasi web
\begin{enumerate}
    \item Sig in ke apex.oracle.com
    \item Membuat workspace
    \item Buka app builder
    \item Create a new app
    \item Drag dan drop file
    \item Pilih movie, next
    \item Isi pada table nama, serta jenis huruf yang digunakan
    \item Setelah itu create application
    \item Masuk appreance yang dimana berisikan jenis jenis table yang ada
    \item Pilih icon sesuai keinginan
    \item Pada feature centang semua kolom yg tersedia
    \item Run aplikasi dengan memasukan kata sandi dan username
\end{enumerate}

\end{document}
