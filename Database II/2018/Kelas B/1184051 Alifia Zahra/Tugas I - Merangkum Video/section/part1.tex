\clearpage
\setcounter{page}{1}

\begin{center}
\title{\LARGE \bf Resume Video}
\end{center}

\section*{\normalsize 1. Oracle Apex} 
\hspace {\parindent}Oracle apex merupakan suatu platform pengembangan kode rendah yang membangun aplikasi perusahaan. Bekerja sebagai lingkungan pengembangan software berbasis web yang berjalan/berbasis pada database.
\begin{enumerate}[label=\alph*.]
\item Adapun langkah-langkah membuat Workspace di APEX adalah\\
1. Install APEX terlebih dahulu.\\
2. Buka http://127.0.0.1:8080/apex/f?p=4550:10:10960410872195:::::\\
3. Login dengan akun yang sudah dibuat ketika anda meng-install APEX.\\
4. Lalu anda akan memasuki halaman utama.\\
5. Setelah itu klik Create Workspace\\
6. Buatlah nama dari workspace anda.\\
7. Klik next untuk melanjutkan.\\
8. Untuk mengubah skema nya menjadi HR, klik di samping toolbar ADMIN maka akan keluar pop up berisikan HR\\
9. Setelah itu klik next.\\
10. Atur username dan password anda untuk login.\\
11. Atur email anda (disarankan yang aktif).\\
12. Klik next\\
13. Akan ada konfirmasi yang muncul mengenai workspace.\\
14. Klik create workspace\\
15. Jika sudah berhasil, maka selanjutnya silahkan menuju ke http://127.0.0.1:8080/apex/f?p=4550:1:12275980854321:::::\\
16. Lalu login dengan workspace dan username serta password yang telah anda buat sebelumnya\\
17. Setelah login berhasil, maka akan ada disarankan untuk mengubah/memperbaharui password.\\
18. Setelah semua langkah telah selesai maka anda akan di bawa ke halaman Workspace. Selesai\\
\end{enumerate}

\section*{\normalsize 2. Pengertian Spreadsheet}
\hspace{\parindent}Spreadsheet berfungsi untuk menyimpan/menampung informasi. Lalu setiap kolom yang berada pada spreadsheet menampung informasi yang berbeda-beda.
\begin{enumerate}[label=\alph*.]
\item Membuat Aplikasi pada Spreadsheet\\
	1. Pergi menuju Http://apex.oracle.com\\
	2. Klik Get Started For Free untuk memulai\\
	3. Klik permintaan ruang kerja yang kosong.\\
	4. Masuk ke ruang kerja tersebut di Http://apex.oracle.com\\
	5. Klik pembuat aplikasi 
	6. Klik buat aplikasi baru\\

\item Mengubah spreadsheet menjadi aplikasi web\\
	1. Silahkan sign in terlebih dahulu pada apex.oracle.com\\
	2. Lalu buatlah suatu workspace\\
	3. Buka app builder\\ 
	4. Create a new app\\ 
	5. Drag dan drop file\\
	6. Pilih movie, lalu klik next\\
	7. Silahkan isi table nama, serta pilih jenis huruf/font yang digunakan\\
	8. Lalu create application\\
	9. Masuk apperance yang  berisikan jenis- jenis tabel yang ada\\
	10. Pilih icon sesuai keinginan anda \\ 
	11. Centang semua kolom pada feature\\
	12. Run aplikasi dengan memasukan kata sandi dan username yang anda miliki\\
\end{enumerate}

\section*{\normalsize 3. Pengertian Oracle Apex Use Case}
\paragraph{} Use Case digunakan untuk memodernisasi bentuk dari suatu oracle. Adapun fitur-fitur dalam usecase yaitu:
\begin{enumerate}
    \item Melakukan drag dan meletakkan file XLS, CSV, XML, atau JSON
    \item Membuat tabel dalam database Otonom
    \item Mengunggah data ke dalam tabel baru
    \item Membuat suatu aplikasi berdasarkan tabel net
\end{enumerate}
\paragraph{} Adapun solusi dalam use case apex adalah:
\begin{enumerate}
    \item Satu sumber jebakan
    \item Mengirim URL bukan file
    \item Scure, scalable, aplikasi multiuser
    \item Melanjutkan dengan obrolan, kalender, validasi, dll.
\end{enumerate}
