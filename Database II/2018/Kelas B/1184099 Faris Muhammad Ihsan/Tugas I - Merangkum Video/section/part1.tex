\clearpage
\setcounter{page}{1}

\begin{center}
\title{\LARGE \bf Rangkuman Video }
\end{center}

\section{Oracle Apex}
	\begin{enumerate}

	\item Oracle Apex adalah platform untuk mengembangkan kode untuk membangun database dari aplikasi perusahaan, dapat 		diperhitungkan ketahanan datanya dan  terjamin keamannya dengan fitur yang tinggi dan dapat digunakan dari mana saja. Oracle apex merupakan Development Environtment yang berbasis web.
	
	\item Membuat workspace
		\begin{enumerate}[label=\alph*.]
		\item Pastikan anda sudah menginstall APEX pada PC atau Laptop.
		\item Buka http://127.0.0.1:8080/apex/
		\item Login dengan akun yang sudah dibuat pada saat menginstall APEX
		\item Masuk ke halaman utama
		\item Klik Create Workspace untuk membuat workspace
		\item Buat nama workspacenya, bagian isian lain boleh dikosongkan atau diisi
		\item Selanjutnya buat pengaturan
		\item Untuk mengubah skema menjadi HR, klik ADMIN sehingga muncul pop up
		\item Atur nama username dan password untuk login 
		\item Atur email dengan email kita
		\item Setelah itu, konfirmasi kembali workspacenya
		\item Kemudian buat workspace
		\item Jika sudah maka login kembali ke http://127.0.0.1:8080/apex/
		\item login sesuai nama dan workspace yang telah dibuat
		\item Jika sudah selesai maka akan diarahkan ke halaman workspace
		\end{enumerate}
	
	\item Spreadsheet\\
	Spreadsheet digunakan untuk pengguna sebagai tempat untuk menyimpan berbagai informasi yang sangat lengkap. Setiap kolom dari spreadsheet dapat menyimpan data data dan informasi yang dibutuhkan oleh penggunanya. Aplikasi form spreadsheet ini berbentuk menjadi beberapa project dan nama, serta keterangan keterangan lainnya. Contohnya seperti Tanggal mulai, tanggal selesai, statis, biaya, dan lainnya. 

	\item membuat aplikasi menggunakan Spreadsheet
		\begin{enumerate}[label=\alph*.]
		\item Pergi ke 
		\item Klik get started for free
		\item klik permintaan ruang kerja yang kosong
		\item masuk ke workspace 
		\item Kemudian buat aplikasi baru
		\end{enumerate}
	
	\item Mengubah Spreadsheet menjadi aplikasi web
		\begin{enumerate}
		\item Log In ke apex.oracle.com
		\item Buat workspace
		\item Buka app builder
		\item buat aplikasi baru
		\item drag filenya
		\item pilih movie
		\item Isi table nama, serta jenis huruf yang digunakan
		\item Create application lalu anda akan memasuki tampilan yang berisi jenis jenis tabel yang ada
		\item pilih icon sesuai keinginan
		\item centang fitur semua kolom
		\item Run aplikasi dengan memasukan password dan username
		\end{enumerate}
	
	\end{enumerate}