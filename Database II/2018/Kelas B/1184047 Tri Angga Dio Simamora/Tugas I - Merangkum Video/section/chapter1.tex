\chapter{Mengenal APEX Oracle}

\section{Agenda Kegiatan APEX}

\subsection{Aplikasi Low Code}
Low code adalah pengembangan dari aplikasi berbasis \textit{Graphical User Interfaces} (GUI), sehingga memungkinkan untuk membuat/mengembangkan sebuah aplikasi dengan sedikit kodingan bahkan hingga tidak memakai kodingan sama sekali.

\begin{enumerate}
\item Low code sangat mudah digunakan dikarenakan aplikasinya berbasis GUI.
\item Fitur aplikasi low code sangat banyak sehingga memungkinkan user bisa lebih produktif dalam menghasilkan sebuah aplikasi
\item Setelah aplikasi di buat kita dapat langsung memakainya tanpa harus mengkoding lagi.
\item Low code juga sangat mudah dalam maintenance nya.
\item Fungsionalitas low code yang banyak sekali sehingga program dapat menggunakan fitur yang melimpah.
\end{enumerate}

\subsection{Penjelasan Oracle APEX}
APEX merupakan kerangka kerja pengembangan aplikasi dan database yang terpusat. Fungsi dan kegunaan APEX antara lain.
\begin{enumerate}
\item Membangun aplikasi mulai dari desktop, mobile, dan web app
\item Memvisualisasikan dan memelihara data pada database
\item Meningkatkan kemampuan SQL dan kemampuan basis data
\end{enumerate}

\subsection{Mengubah dari Spreadsheet menjadi Sebuah Aplikasi Web dalam Hitungan Menit}
Ketika kita mengelola data pada spreadsheet tantangan yang harus kita lalui yaitu validasi data. Integritas data pada spreadsheet tidak dapat dijamin apakah data tersebut tidak redudansi. Keamanan data harus diperhatikan dikarenakan google spreadsheet memiliki fitur share sehingga orang yang memiliki akses harus berhati-hati karena share tersebut sangat mudah dilakukan.

\subsection{Kegunaan APEX}
\begin{enumerate}
\item Sebagai pengganti spreadsheet APEX Oracle menyediakan fitur drag and drop atau upload file yang berekstensi xls, csv, xls, xml, atau json file. Sehingga APEX akan mentranslasikan sebuah data yang berada pada file tersebut menjadi sebuah tabel, jadi 1 file akan menjadi 1 tabel.
\item Fitur yang dimiliki oracale yaitu form yang natural, tidak terlalu rumit seperti APEX ini support SQL maupun PL/SQL, dan juga APEX ini dapat menggunakan kembali paket database, prosedur, dan fungsinya yang telah kita buat sebelumnya. Sehingga developer yang lain dapat mengembangkan aplikasinya secara bebas dan juga dapat dikembangkan dengan mudah
\item Dengan APEX kita bisa membuat aplikasi dalam hitungan hari atau minggu, tidak membutuhkan waktu yang lama hingga bertahun-tahun. Jadi seakan-akan kita sebagai penyihir yang dapat menciptakan sebuah aplikasi dalam hitungan hari atau minggu dengan memiliki fitur-fitur yang kompleks.
\item Bisa membangun aplikasi dalam untuk skala perusahaan yang besar, APEX ini juga bisa membangun aplikasi yang besar dan kompleks dalam waktu yang singkat sehingga melaporkan dan memperbaiki masalah pada data perusahaan dapat lebih cepat dilakukan, dan memungkinkan kita untuk memisahkan data silo yang berada pada perusahaan.
\item Memperluas sistem perusahaan baik internal maupun eksternal sehingga perusahaan memiliki waktu yang luas untuk mengembangkan aplikasinya secara luas.
\end{enumerate}

\subsection{Aplikasi Produktivitas}
Selain membuat aplikasi dari spreadsheet dengan APEX kita juga bisa membuat aplikasi dengan fitur Productivity App atau aplikasi produktivitas yang didalamnya terdapat fitur yang dapat kita gunakan sebagai aplikasi kita. Berikut langkah yang dilakukan untuk membuat aplikasi produktivitas.
\begin{enumerate}
\item Klik app build, kemudian pilih Productivity App
\item Kemudian pilih aplikasi yang akan dibuat, kita ambil Sample Charts
\item Klik install app
\item Kemudian pilih next
\item Klik install app
\item Tunggu hingga proses instalasi selesai
\item Jika sudah selesai maka jalankan aplikasi
\item Lalu, login menggunakan akun oracle masing-masing
\end{enumerate}