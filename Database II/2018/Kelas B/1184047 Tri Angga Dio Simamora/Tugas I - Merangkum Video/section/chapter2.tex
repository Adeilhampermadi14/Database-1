\chapter{Low-code Development Application}
Low-code merupakan sebuah software yang mana berisi sebuah aplikasi-aplikasi siap pakai dari software tersebut akan membuat sebuah aplikasi baru yang pembuatannya berbasis graphical user interface dan beberapa konfigurasi pada aplikasi yang ingin kita bangun.

\section{Mudah Digunakan}
Low-code ini sangat mudah digunakan dikarenakan berbasis \textit{GUI} sehingga orang yang ingin membuat aplikasi baru seperti website misalnya, mereka hanya memasukkan parameter yang mereka butuhkan tanpa berhubungan dengan \textit{coding}.

\section{Sangat Produktif}
Aplikasi yang disediakan pada platform berbasis low-code mempunyai banyak fitur sehingga produktivitas anda dalam membuat aplikasi akan meningkat dan membuat aktivitas aplikasi semakin kompleks dan semakin produktif.

\section{Skalabilitas yang Tinggi}
Aplikasi yang dihasilkan denga low-code akan menjadi aplikasi yang mempunyai skala yang luas, sehingga program tersebut dapat di\textit{maintenance} secara mudah, dan ketika ada perubahan pada aplikasi tersebut, aplikasi itu siap untuk dirubah menjadi sedemikian rupa sesuai keinginan user.

\section{Ekstensibilitas yang Tinggi}
Tidak hanya memiliki skala yang tinggi, aplikasi tersebut akan menunjang dengan seiring perkembangan zaman dan teknologi dalam aplikasi, sehingga aplikasi yang kita hasilkan mempunyai fitur-fitur terbaru.

\section{Fungsi yang Melimpah tetapi Sedikit dalam Menulis Code}
Aplikasi yang kita buat adalah berbasis \textit{GUI} sehingga kita akan menemukan sedikit sekali dalam \textit{coding} mungkin hampir tidak ada sama sekali, dan juga platform low-code menyediakan fungsi yang banyak sehingga orang yang membuat aplikasi tersebut tinggal menambahkan parameter-parameternya.m