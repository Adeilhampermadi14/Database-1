\clearpage
\setcounter{page}{1}

\begin{center}
\title{\LARGE \bf Rangkuman Video}
\end{center}

\section*{\normalsize 1. Oracle Apex} 
\hspace {\parindent}Oracle apex adalah platform pengembangan kode rendah yang membangun aplikasi perusahaan. Oracle apex juga merupakan lingkungan pengembangan perangkat lunak berbasis web yang berjalan pada database.
\begin{enumerate}[label=\alph*.]
\item Cara membuat Workspace di APEX\\
1. Install APEX .\\
2. Buka http://127.0.0.1:8080/apex/f?p=4550:10:10960410872195:::::\\
3. Login dengan akun yang sudah dibuat waktu install APEX.\\
4. Jika sudah, maka akan masuk ke halaman utama.\\
5. Setelah itu klik Create Workspace\\
6. Buat nama dari workspace nya.\\
7. Klik next\\
8. Untuk mengubah skema nya menjadi HR, maka klik di samping ADMIN, nanti akan ada pop up berisi HR.\\
9. Setelah itu klik next.\\
10. Atur username dan password untuk login.\\
11. Atur email.\\
12. Klik next\\
13. Setelah next, akan ada konfirmasi kembali mengenai workspace.\\
14. Klik create workspace\\
15. Jika sudah berhasil, maka selanjutnya pergi ke http://127.0.0.1:8080/apex/f?p=4550:1:12275980854321:::::\\
16. Maka Login sesuai dengan nama workspace dan username password tadi.\\
17. Setelah login, maka akan ada disuruh ubah password.\\
18. Jika sudah selesai, maka akan langsung menuju hal workspace\\
\end{enumerate}

\section*{\normalsize 3. Pengertian Spreadsheet}
\hspace{\parindent}Spreadsheet digunakan untuk menyimpan berbagai informasi. Disetiap kolomnya bisa menyimpan berbagai informasi yang berbeda.
\begin{enumerate}[label=\alph*.]
\item Membuat Aplikasi dari Spreadsheet\\
	1. Pergi ke Http://apex.oracle.com\\
	2. Klik Get Started For Free\\
	3. Klik permintaan ruang kerja yang kosong.\\
	4. Masuk ke ruang kerja di Http://apex.oracle.com\\
	5. Klik pembuat aplikasi 
	6. Klik buat aplikasi baru\\

\item Mengubah spreadsheet menjadi aplikasi web\\
	1. Sign in apex.oracle.com\\
	2. Membuat workspace\\
	3. Buka app builder\\ 
	4. Create a new app\\ 
	5. Drag dan drop file\\
	6. Pilih movie, next\\
	7. Isi pada table nama, serta jenis huruf yang digunakan\\
	8. Setelah itu create application\\
	9. Masuk appreance yang dimana berisikan jenis- jenis tabel yang ada\\
	10. Pilih icon sesuai 	keinginan\\ 
	11. Pada feature centang semua kolom\\
	12. Run aplikasi dengan memasukan kata sandi dan username\\
\end{enumerate}