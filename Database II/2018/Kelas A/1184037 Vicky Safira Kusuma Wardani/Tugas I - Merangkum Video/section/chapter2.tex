\chapter{Tinjauan Pustaka}


\section{Restrict/memabatasi status}
\begin{enumerate}
    \item Untuk memilih tipe Select List ada pada halaman designer dalam property editor dan klik kanan terlebih dahulu
\item Lalu Memilih tipe SQL Query pada list of values dan klik code editor
\item Ketik tulisan berikut ini di dalam code editor:
select distinct status d, status r 
from spreadsheet
order by 1
\item	Lalu klik validate dan oke jika selesai.
\item	Pilih No pada Display Extra Values dan ketik -Select Status- pada Null Value Display
\item	Setelah itu klik simpan.

\end{enumerate}


\section{Menjalankan Aplikasi}
\par Untuk menjalankan aplikasi, arahkan lagi ke runtime environment, lalu refresh browser yang digunakan, dan edit record. Klik status dan ubah menjadi closed.



 \section{Link Calender Untuk Meng-Update Form}
 \begin{enumerate}
     \item 	Pada page designer, klik attributes dibawah calender
 \item 	Klik view/edit link
 \item 	Di atribut page, pilih 3.
 \item 	Atur Clear Cache menjadi 3, dan klik OK.
 \item 	Lalu, klik save and run. Kalender sudah berjalan.

 \end{enumerate}


