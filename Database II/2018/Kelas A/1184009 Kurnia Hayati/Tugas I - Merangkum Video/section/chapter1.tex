\chapter{}

\section{Tutorial Membuat Aplikasi Akademik di Apex}

\section{Mengelola Data Dalam Spreadsheet Adalah Chanenging}
\begin{enumerate}
    \item 1.	Ketik di google apex online lalu create workspace dan tampilannya akan seperti ini lalu next
    
    \item 

\end{enumerate}


\section{Aplikasi Yang Cocok Untuk Oracle Apex}
\begin{enumerate}
\item Microsoft Acces
\item Oracle
\item Ms SQL Server
\item MySQL
\item Firebird
\item Postsgre SQL
\item mengganti spreadsheet
\end{enumerate}

\section{Oracle Apex Membedakan Karakteristik}
\begin{enumerate}
\item pengembangan aplikasi IDE adalah browser web, tidak perlu perangkat lunak klien
\item definisi aplikasi disimpan dalam database sebagai data meta, deklaratif tanpa \item pembuatan kode
\item pemrosesan data dilakukan dalam database

\end{enumerate}

\section{Contoh Database Tunggal / Beberapa Ruang Kerja}
\begin{enumerate}
    \item ruang kerja yang digunakan untuk mendefinisikan definisi aplikasi / skema menyimpan data
  \item banyak ke banyak hubungan antara ruang kerja dan skema
  \item administrator contoh mengelola lingkungan dan akses skema
  \item dapartments dapat meminta lebih banyak ruang, dan akses ke skema baru
 
\end{enumerate}