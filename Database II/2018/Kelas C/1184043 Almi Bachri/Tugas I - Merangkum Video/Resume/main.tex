\documentclass{article}
\usepackage[utf8]{inputenc}

\title{Resume Seminar Online Oracle}
\author{Almi Bachri (1184043) }
\date{Oktober 2019}

\begin{document}

\maketitle

\section{Apa Itu Oracle Apex?}
Oracle APEX adalah platform pengembangan kode rendah yang memungkinkan Anda membangun aplikasi perusahaan yang dapat diskalakan dan aman dengan fitur kelas dunia yang dapat digunakan di mana saja


\section{Cara Membuat Database Baru}
\begin{enumerate}
    \item Login Application Express 
    \item Pilih App Builder
    \item Pilih Create an application
    \item Pilih From a file
    \item Pilih file yang akan kalian buat table database baru
\end{enumerate}
\section{Cara Menggunakan Product Yang ada di Aplikasi}
\begin{enumerate}
    \item Login Application Express
    \item Pilih new application
    \item Pilih App gallery
    \item Pilih sample
    \item Lalu install Aplikasi yang kalia ingin jalankan
    \item Excel
\end{enumerate}
\section{Cara Mencari Data Yang Sudah Dibuat}
\begin{enumerate}
    \item Pilih SQL Workshop 
    \item Pilih SQL Commands
    \item Masukkan perintah select * from nama data yang sudah dibuat
\end{enumerate}
\section{Pengertian Spreadsheet}
\usepackage {Spreadsheet: Memungkinkan Pengguna Untuk Menyimpan Berbagai Informasi Yang Sangat Lengkap, Pada Setiap Kolomnya Bisa Menyimpan Berbagai Data Informasi Yang Berbeda Dari Informasi Yang Di Perlukan.App From Spread- sheet Disini Berupa Beberapa Project Dan Nama Tugas Nya Serta Keterangan Lainnya Seperti Tanggal Mulai, Tanggal Selesai, Status, Di Ttd Oleh,Biaya, Budget Tersedia, Dan Lebih Kurangnya Dari Budget.}
\section{Cara Membuat Spreedsheet}
\begin{enumerate}
    \item Login Oracle APEX
    \item Klik App Builder
    \item Klik a new app
    \item Klik from a file
    \item Klik Copy and Paste
    \item Untuk data sample set select Project and Tasks
    \item Masukkan nama tabel spreedsheet
    \item Klik Load Data
    \item Cek kolom 73
    \item Klik Continue to Create Application Wizard 
    \item Masukkan nama (app dari spreedsheet)
    \item Lanjut ke features, klik Check All
    \item Klik Create Application
    \item App yang kamu buat akan muncul di page designer 
    \item Klik Run Application
\end{enumerate}

Oracle Apex: Use Cases
\section{Fitur}
\begin{enumerate}
    \item Drag Dan Letakkan File Xls, Csv, Xml, Atau Json
    \item Membuat Tabel Dalam Database Otonom
    \item Unggah Data Ke Tabel Baru
    \item Buat Aplikasi Berdasarkan Tabel Baru
    \item Meluas Erps Dan Perangkat Lunak Perusahaan Lainnya
    \item Menyediakan Dasbor Khusus Organisasi
    \item Alur Kerja Yang Lebih Baik
    \item Mengisi Kekosongan
    \item Fitur Tanpa Biaya Dari Database Oracle
\end{enumerate}
\section{Mengurutkan Laporan Interaktif}
\begin{enumerate}
    \item Klik Spreedsheet
    \item Klik Action, Select Data, Select sort
    \item for 1, start date, for 2, select and date, click Apply
\end{enumerate}

\section{ Menambahkan Komputasi}
\begin{enumerate}
    \item Klik Actions, Pilih Data, Pilih Computer
    \item column Label Masuk Bugget V Cost
    \item Format Mask pilih 5,243,10
    \item Masukkan Ekspresi Komputasi I-H
    \item klik apply
\end{enumerate}
\section{ Menambahkan Sebuah Grafik}
\begin{enumerate}
    \item Klik “Action”, Dan Pilih “Chart” 
    \item Label Pilih “**Budget V Cost”
    \item Fungsi Pilih “ Sum”
    \item Sort Pilih “Label-Ascending”
    \item Orientasi Pilih “Horizontal”
    \item Klik “Apply”//
    *NB:Untuk Pengeditan Grafik Bias Dilakukan Dilaman App From Spreadsheet
\end{enumerate}
\section{Menyimpan Laporan}
\begin{enumerate}
    \item Klik”Action”,Pilih “Report”, Pilih”Save Report”
    \item Untuk Simpan, Pilih “As Default Report Settings”
    \item Tipe Default Laporan, Pilih “ Alternative”
    \item Nama, Enter “Data Review”
    \item Klik “Apply”
\end{enumerate}
\section{ Batasi Status}
\begin{enumerate}
    \item Ketika Runtime Environment, Klik “Edit Icon On A Record”
    \item Halaman Modal Akan Tampil
    \item Pada Developer Toolbar, Klik “Quick Edit”
    \item Pada Status Item (Tunggu Sampai Outline Biru Muncul” Lalu Klik Mouse
    \item Di Halaman Designer Muncul Dengan Focus Pada Status Item
    \item Di Halaman Designer, Dalam Editor Property(Panel Kanan)
    \item Di Bawah Daftar Nilai-Nilai, Untuk Type Pilih “SQL Query”
    \item Lanjutkan Ke SQL Query, Klik “Code Editor”
\end{enumerate}
\section{ ”Membatasi Status (Restrict The Status”}
\usepackage {Dalam Kode Masukkan Seperti Ini}
\begin{enumerate}
    \item Select Distinct Status D, Status R From Spreadsheet Order By 1
    \item Klik Validate
    \item Klik Ok
    \item Untuk Menampilkan Nilai Ekstra Pilih No
    \item Menampilkan Nilai Null Pilih –Select Status-
    \item Klik Save (Pada Toolbar Top Right)
\end{enumerate}
\section{ ”Menjalankan Apllikasi (Run Aplikasi)”}
\begin{enumerate}
    \item Arahkan Navigasi Ke Runtime Environment
    \item Refresh Browser
    \item Edit Record
    \item Klik Status
\end{enumerate}
\section{ Tambah Kalender (Add A Calender)}
\begin{enumerate}
    \item Arahkan Navigasi Kembali Ke Development Environtment
    \item Pada Aplikasi Builder, Arahkan Pada Home Page
    \item Klik Create Page
\end{enumerate}
\section{Langkah 4.1a}
\begin{enumerate}
    \item Klik Pada Calender
    \item Pada Page Name Pilih Breadcrumb
    \item Lalu Klik Next
\end{enumerate}
\section{ Tambahkan Kalender}
\begin{enumerate}
    \item Klik Kalender
    \item Nomor Halaman, Masukkan Kalender
    \item Remah Roti, Pilih Remah Roti
    \item Klik Selanjutnya
\end{enumerate}
\section{ Tambahkan Kalender}
\begin{enumerate}
    \item Preferensi Navigasi, Klik Buat Entri Menu Navigasi Baru
    \item Klik Selanjutnya
    \item Tabel / Nama Tampilan, Pilih Spreadsheet (Tabel)
    \item Klik Selanjutnya
\end{enumerate}
\section{ Tambahkan Kalender}
\begin{enumerate}
    \item Kolom Tampilan, Pilih Task Nama Kolom tanggal Akhir, Pilihan Date
    \item Klik Buat
\end{enumerate}
\section{ Menautkan Kalender Ke Dari}
\begin{enumerate}
    \item Di Tab Rendering, Di Bawah Kalender, Klik Atribut
    \item Di Editor Properti (Panel Kanan), Klik Lihat / Edit Tautan
    \item Halaman, Pilih 3
    \item Mengatur Item Nama, Pilih Pada Nilai, Pilih ID Bersihkan Cache, Masukkan3
    \item Klik Ok
    \item Klik Simpan Dan Jalankan
\end{enumerate}

\section{Membuat Aplikasi Dari Spreadsheet}
\subsection{Langkah 1.1 a}
\begin{enumerate}
    \item Pergi Ke Http://Apex.Oracle.Com
    \item Klik Get Started For Free
\end{enumerate}
\subsection{Langkah 1.1 b}
\begin{enumerate}
    \item Klik Permintaan Ruang Kerja Yang Kosong. 
\end{enumerate}
\section{Langkah 4.2b : Menautkan Kalender Ke For- mulir Pembaruan}
\usepackage {Note: Anda Mungkin Harus Menavigasi Ke Bulan Mei Untuk Melihat Entri Kalender}
\section{Tautan Yang Bermanfaat}
\begin{enumerate}
    \item Apex Collateral Http://Apex.Oracle.Com
    \item Tutorial Http://Apex.Oracle.Com/En/Learn/Tutorial
    \item Community External Site + Slack Http://Apex.Oracle.Com/Community
\end{enumerate}
\end{document}

