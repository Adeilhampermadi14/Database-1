\documentclass{article}
\usepackage[utf8]{inputenc}

\title{Tugas}
\author{yukiardiansyah321 }
\date{October 2019}

\begin{document}

\maketitle


\section{Oracle APEX}
\usepackage{Oracle Apex Adalah Aplikais Yang Digunakan Oleh Pelanggan Nyata Untuk Aplikasi Nyata Yang Digunakan Untuk Aplikasi Kritis Oportuninistik Dan Misi Yang Melayani Puluhan Ribu Pengguna Produk Mapan Pertama Kali Dirilis Pada Tahun 2004 Platform Pengembangan Aplikasi Kode Rendah Yang Paling Kuat: Memungkinkan Pengembangan Untuk Ficus Dalam Memecahkan Masalah Bisnis Dan Memberikan Solusi Yang Unggul, Dengan Lebih Sedikit Waktu Dan Upaya Yang Dihabiskan Untuk Pengodean Tingkat Rendah Biasa Dan Berulang Terus Berkembang. Oracle Install Base Mengadopsi Oracle Apex Untuk Meningkatkan Jumlah Proyek Dan Semakin Menjadi Standar IT Korporat Yang Disetujui
}

\section{Pengertian Spreadsheet}

\usepackage {Spreadsheet: Memungkinkan Pengguna Untuk Menyimpan Berbagai Informasi Yang Sangat Lengkap, Pada Setiap Kolomnya Bisa Menyimpan Berbagai Data Informasi Yang Berbeda Dari Informasi Yang Di Perlukan.App From Spreadsheet Disini Berupa Beberapa Project Dan Nama Tugas Nya Serta Keterangan Lainnya Seperti Tanggal Mulai, Tanggal Selesai, Status, Di Ttd Oleh,Biaya, Budget Tersedia, Dan Lebih Kurangnya Dari Budget.}

\section {Pendidikan}
\usepackage{Apakah Anda Seorang Siswa Atau Guru SQL, Database Relasional, atau Pengembangan Aplikasi, Anda Dapat Menggunakan Oracle Apex Untuk Sangat Memperkaya Pengalaman Pendidikan Anda?}

\section {Sertifikasi APEX}
\usepackage{Setelah Anda Mahir Mengembangkan Aplikasi APEX, Anda Dapat Mengikuti Ujian Sertifikasi Oracle Menjadi Aplikasi Oracle Express 18: Profesional Bersertifikat Pengembang.Menonjol Di Antara Rekan-Rekan Anda, Dan Buktikan Kepada Semua Orang Bahwa Anda Tahu Cara Membangun Aplikasi Yang Kuat Dengan Menggunakan Apex.}

\section {Kurikulum Gratisan Oracle Apex}
\begin{enumerate}
    \item Pelajar, Dan Panduan Praktikum Di Laboratorium
    \item Total 16 Pelajaran Dan 15 Tangan Di Laboratorium
    \item PPT, PDF, Sumber, Dan File Lab
    \item Lab / Demo Dapat Dilakukan Pada: Contoh Akademi Oracle

\end{enumerate}

\section {Gambaran Umum }
\usepackage {Lab Ini Menuntun Anda Saat Mengunggah Spreadsheet Ke Tabel Database Oracle, Lalu Membuat Aplikasi Berdasarkan Tabel Baru Ini.  Anda Kemudian Akan Bermain Dengan Laporan Interaktif Dan Meningkatkan Formulir Terlampir.  Terakhir, Anda Akan Menambahkan Halaman Kalender Dan Kemudian Menautkannya Ke Halaman Formulir Yang Ada.  Alih-Alih Mencoba Mengirim Surel Spreadsheet Untuk Mengumpulkan Informasi Dari Orang Yang Berbeda, Cukup Buat Aplikasi Dalam Hitungan Menit, Dan Kirim Surel URL.  Spreadsheet Sumber-Kebenaran-Tunggal, Multi-Pengguna, Aman, Dan Mudah Tersiram Ini!  Aplikasi Scren Jadi Lebih Baik}

\section{Apa Itu Oracle Apex?}

    \usepackage {Oracle Application Express (Oracle APEX) yang dulu disebut HTML-DB adalah sebuah framework yang berbasis pada sebuah database dedicated (sementara ini sampai versi terbaru masih dedicated untuk Oracle Db saja dan lisensi include dalam lisensi database), ini artinya apa bahwa engine aplikasi dibangun sepenuhnya didalam sebuah database. Bahkan untuk arsitektur Embedded PL/SQL Gateway seperti yang dipakai dalam Oracle XE dan Oracle 11G file image (library,css,theme,dll) disimpan didalam database metadata juga. Inilah hal yang berbeda dibandingkan framework yang lain.}gf


\end{document}
