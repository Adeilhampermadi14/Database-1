\documentclass{article}
\usepackage[utf8]{inputenc}

\title{Basis data 2}
\author{ryanface290300 }
\date{October 2019}

\begin{document}

\maketitle

\section{Oracle Apex Untuk Guru}
  \begin{enumerate}
      \item SQL dan Database Relasional 
      \item Tangan di laboratorium dengan sql Pengembangan aplikasi 
      \item Visualisasi data 
      \item Dapatkan Mulai secara gratis di https://apex.oracle.com atau gunakan \item Oracle Academy Apex Instance 
      \item  Mengatur Akun Siswa

  \end{enumerate}
\section{Kurikulum Oracle APEX}
   \begin{enumerate}
       \item SQL dan Database Relasional 
       \item Tangan di laboratorium dengan sql Pengembangan aplikasi 
       \item Visualisasi data 
       \item Dapatkan Mulai secara gratis di https://apex.oracle.com atau gunakan Oracle Academy Apex Instance 
       \item Mengatur Akun Siswa

   \end{enumerate}
  \section{Lab Praktek}
     \begin{enumerate}
         \item Membuat Aplikasi dari Spreadsheet
              https://apex.oracle.com/en/learn/tutorials
     \end{enumerate}
\section{Sertifikasi Oracle Apex}
\subsection{Langkah 1.1}
      \begin{enumerate}
          \item Perikssa Email anda, anda harus mendapatkan email dari oracle.application-express www@oracle.com dalam beberapa menit
          \item Klik Buat Ruang Kerj
          \item Klik lanjutkan ke layar masuk
          \item Mereset Password
    
      \end{enumerate}
\subsection{Langkah1.2-Masuk}
     \begin{enumerate}
         \item . Masuk ke ruang kerja Anda di https://apec.oracle.com
         \item  Klik Pembuat Aplikasi
         \item  Klik Buat Aplikasi Baru

     \end{enumerate}

\subsection{Langkah 1.3-Memilih Jenis Aplikasi}
      \begin{enumerate}
          \item  Klik Dari File
        \item Klik salin dan tempel
          \item Untuk sampel Kumpulan Data pilih
          Proyek dan Tugas
    \end{enumerate}
    
\subsection{Langkah 1.4-Penanaman Jenis Aplikasi}
      \begin{enumerate}
          \item Masukan Nama(Aplikasi dari Spreadsheet)
          \item Disebelah fitur,klik periksa semua
      \end{enumerate}
\subsection{Langkah 1.5-App Di halaman Designer}
    \begin{enumerate}
        \item Aplikasi Baru and akan ditampilkan di halaman designer
        \item Klik Jalankan Aplikasi
    \end{enumerate}
    
\subsection{Langkah 1.6-Aplikasi Runtime}
     \begin{enumerate}
         \item Masukan Kredensial Pengguna Anda
         \item Masukan dengan Nama Aplikasi Anda
     \end{enumerate}
\subsection{Langkah 1.7-Tambahkan grafik}
      \begin{enumerate}
          \item Klik Tindakan, bagan select 
          \item Label pilih Project 
           \item Nilai pilih Jumlah 
           \item Fungsi Pilih Jumlah
          \item Orientation pilih Horizontal
          \item  Klik Terapkan
      \end{enumerate}
\subsection{Langkah 1.8-Simpan Laporan}
\begin{enumerate}
    \item Klik Aksi,pilih laporan,pilih simpan Laporan
    \item Untuk simpan,pilih sebagai pengaturan laporan default jenis laporan kerusakan,pilih alternatif
    \item Nam,masukan Tanggl Review
    \item Klik Terapkan
\end{enumerate}
\subsection{Langakah 1.9-Batasi Status A}
    \begin{enumerate}
        \item Dalam Desaigner halaman editor properti(panel kanan) Select List
        \item Dibawah daftar nilai untuk tipe pilih SQL Query,Klik Editor
    
    \end{enumerate}
\subsection{Langkah 2.0- Batasi Status B}
    \begin{enumerate}
        \item Dengan Editor Kode,masukan yang berikut ini:
        \item *Pilih status d berbeda,status r dari urutan spredsheet oleh 1
        \item Klik Validasi
        \item Klik Ok
        \item Menampilkan nilai tambaahan pilih tidak tampilan Nilai-Null,masukan-pilih status
        \item Klik Simpan
        
    \end{enumerate}

\section{Jadwal Acara}
    \begin{enumerate}
        \item Pengembangan Kode Aplikasi Rendah
        \item Ikhtisar Oracle Aplikasi
        \item Mengubah Spreadsheet menjadi aplikasi web dalam hitungan menit-demo
        \item Oracle Apex Education
        \item Laboratorium Praktek
    \end{enumerate}
\subsection{Pengembangan Aplikasi Di Perusahaan}
     \begin{enumerate}
         \item Memerlukan Sumber Daya Pengembangan Khusus yang mahal siklus dev apliksi lama
         \item Backlog Yang Besar
         \item Kolaborasi Minimal
         \item Bisnis memecahkan masalah dengan alat yang salah
     \end{enumerate}
\section{Apa itu Kode Rendah?}
      \begin{enumerate}
          \item Mudah dijalankan
          \item Super Produktif
          \item Dapat Diukur
          \item Dapat Diperpanjang
          \item Fungsionalitas yang kaya dengan kode yang sedikit
      \end{enumerate}
\section{Mengelola Data dalam Spreadsheet yang Menantang}
      \begin{enumerate}
          \item Data Validasi-Manualdan rawan kesalahan
          \item Data Integritas Tidak dapat menjamin keakuratan data dalam lingkungan multi-guna
          \item Penguncian Data Keamanan sel-Tidak Aktif
          \item Data Sharing Excel lamban dan sulit untuk dibagikan
          
      \end{enumerate}
\section{Oracle APEX}
      \begin{enumerate}
          \item Aplikasi yang sangat penting bagi ribuan pengguna
          \item Mengisi Gab dalam sistem Perusahaan
          \item Streamline proses bisnis yang sudah ketinggalan zaman
          \item Modernisasi Sistem Warisan
\subsection{Aplikasi layanan diri untuk semua karyawan}
      \begin{enumerate}
          \item Portal yang menghadap pelanggan
          \item Aplikasi responsif yang berfungsi pada perangkat apapun
          \item Bukti Konsep
          \item Aplikasi cepat Menang(umur<beberapa bulan)
          \item Mengganti Spreadsheet
      \end{enumerate}
\section{Oracle Apex Use Case}
  \subsection{Fitur}
  \begin{enumerate}
      \item Apex adalah evolusi alami bentuk keduanya berdasarkan SQL dan PL/SQL
      \item Gunakan kembali paket DB, procedur, fungsi
            Dengan mudah melatih formulir, Pengembang mengembangkan dengan APEX

  \end{enumerate}
 \section{Solusi}
       \begin{enumerate}
           \item Bukti konsep menggunakan subset dari appas froms
           \item Aplikasi .Organisasi, bukan back-office

       \end{enumerate}
\subsection{Aplikasi Eksternal Untuk Pelanggan Mitra}
       \begin{enumerate}
           \item Aplikasi mobile-first
           \item  Persyaratan baru Net
       \end{enumerate}
\subsection{Fitur Tanpa Biaya dari DataBase Oracle}
\section{No fitur Cost yang dideukng sepenuhnya}
    \begin{enumerate}
        \item Sejumlah Aplikasi pengembang & pengguna akhir 
        \item Dukungan Tim dukungan Oracle
        \item 11gR2,12c,18c
        \item Semua Edisi DB:EE,SE2,XE
    \end{enumerate}
\section{Termasuk Layanan Oracle cloud }
\subsection{BasisData Otonom}
    \begin{enumerate}
        \item BasiData Sebagai layanan
        \item Tidak ada evaluasi biaya
    \end{enumerate}
\section{Mudah untuk Menginstall}
 Termasuk secara default dengan semua edisi Database Oracle  
 unduhan unduhn terbaru dari: Berkembag cepat,menyesuaikan,dan memberikan kontrol panel pra bangun untuk keamanan,ontentikasi,interaksi basisdata,validasi managemen sesi,dan banyak lagi...Mulai dari prototype hingga produksi dalam hitungan menit
 \section{Opsi Pengembangan/Penempatan}
  Warga Setempat Install di laptop yang berdiri sebdir menggunakan oracle Express Edition(XE) atau versi DataBase lengkap
      \begin{enumerate}
          \item Cukup Tingkatkan APEX ke versi yang diperlukan
          \item Dapat Bekerja Sepennuhnya
      \end{enumerate}
\subsection{Dilokasi}
      \begin{enumerate}
          \item Biasanya dijalankan oleh Department TI
          \item IT umumnya adalah layanan operasi produksi dan penyedia layanan jasa
      \end{enumerate}
\subsection{Cloud}
      \begin{enumerate}
          \item Mempekerjakan Aplikasi Internet
          \item Levarge untuk pengembangan aplikasi yang cepat penerimaan dan pelatihan pengguna
          \item Prototyping dan Bukti Konsep Perusahan
          \item Berkonsultasi mengembangkan untuk pengembangan pada remis pelanggan
      \end{enumerate}
\section{Mesin Virtual Basis Dat tunggal/Beberapa Ruang Kerja}
Workspace digunakan untuk mendefenisikan definisi aplikasi/schemas menyimpan data.Banyak-ke-Banyak hubungan antara ruang kerja dan skema administrator instansi mengelola lingkungan dan akses skema. Department dapat meminta lebih banyak ruang dan akses ke skema baru Misalnya,layanan hanya-internal oracle memiliki lebih dari 5,000 ruang kerja yang mencakup setiap lini bisini di ORACLE  
      \end{enumerate}
\end{document}
